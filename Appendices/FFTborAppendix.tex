%!TEX root = ../main.tex

\chapter{FFTbor Appendix}
\label{ch:fftborapp}

\lhead{FFTbor Appendix}

\section{Full recursions for \texorpdfstring{\emZ{i,j}}{}
for the Turner energy model}
\label{sec:fftbor:turner}

To compute $\emZ{} = \emZ{1,n}$ given input structure \str, we use the recursions

\begin{align}
\label{eq:fftborapp:zx}
\emZ{i,j} = \emZ{i,j-1} \cdot x^{d_0} +
\sum_{\mathclap{\substack{(s_k,s_j) \in \bpSet, \\ i \leq k < j}}}\enspace
\Big( \emZ{i,k-1} \cdot \emZB{k,j} \cdot \boltzE{E_d} \cdot x^{d_1} \Big),
\end{align}

where $d_0 = 1$ if $j$ is base paired
in $\str_{[i,j]}$ and $0$ otherwise, $d_1 =
\dBP{\str_{[i,j]}}{\str_{[i,k-1]} \cup \str_{[k,j]}}$, and $E_d$
is the energy contribution due to dangling ends (energy
contributions from single bases stacking on adjacent base pairs) and
closing AU base pairs (since a non GC base pair closing a stem has a
destabilizing effect).  The sum is taken over all possible
base pairs $(k,j)$ with $i \leq k < j$.

We compute \emZB{} using the recursion

\begin{align}
\label{eq:fftborapp:zbx}
\begin{split}
\emZB{i,j} &= \boltzE{EH(i,j)} \cdot x^{d_2} \\
&+ \sum_{\mathclap{\substack{(s_k,s_l) \in \bpSet, \\ i < k < l < j }}}\enspace
\emZB{k,l} \cdot \boltzE{EI(i,j,k,l)} \cdot x^{d_3} \\
&+ \sum_{\mathclap{\substack{(s_k,s_l) \in \bpSet, \\ i < k < l < j }}}\enspace
\Big( \emZM{i+1,k-1} \cdot \emZB{k,l} \cdot \boltzE{(a+b+c(j-l-1))}
\cdot x^{d_4} \Big)
\end{split}
\end{align}

where $d_2 = \dBP{\str_{[i,j]}}{\{(i,j)\}}$,
$EH(i,j)$ is the energy of the hairpin loop with closing base
pair $(i,j)$, $EI(i,j,k,l)$ is the energy of the stack, bulge or
interior loop with the closing base pair $(i,j)$ and the interior
base pair $(k,l)$, $d_3 = \dBP{\str_{[i,j]}}{\str_{[k,l]} \cup
\{(i,j)\}}$, and $d_4 = \dBP{\str_{[i,j]}}{\str_{[i+1,k-1]} \cup
\str_{[k,l]} \cup\{(i,j)\}}$. The first term in the
recursion takes care of the case where $(i,j)$ is the only base pair
in $[i,j]$, i.e. $(i,j)$ closes a hairpin loop. The second term
handles the case where there is an interior loop (or a bulge or a
stack) closed by $(i,j)$ and $(k,l)$. The third term takes care of
all the structures where $(i,j)$ closes a multiloop. To reduce
complexity of the algorithm, the interior and bulge loop size can be
limited to a maximum size of $L$ (taken by default to be 30),
by requiring that $l>j-L$ in the above recursion.

The final recursion, for computing \emZM{}, is

\begin{align}
\label{eq:fftborapp:zmx}
\begin{split}
\emZM{i,j} &= \emZM{i,j-1} \cdot \boltzE{c} \cdot x^{d_0} \\
&+ \sum_{\mathclap{\substack{(s_k,s_j) \in \bpSet, \\ i \leq k < j }}}\enspace
\Big( \emZB{k,j} \cdot \boltzE{(b+c(k-i))} \cdot x^{d_5} \\
&+ \emZM{i,k-1} \cdot \emZB{k,j} \cdot \boltzE{b} \cdot x^{d_6} \Big)
\end{split}
\end{align}

where $d_5 = \dBP{\str_{[i,j]}}{\str_{[k,j]}}$ and $d_6
=\dBP{\str_{[i,j]}}{\str_{[i,k-1]} \cup \str_{[k,j]}}$.
Note that since $\emZM{i,j}$ computes the partition function
contribution under the assumption that $[i,j]$ is part of a
multiloop, there will be exactly one stem-loop structure in this
region (the \emZB{} term) or
more than one (the \emZM{}--\emZB{} term).
Justification of recursions (\ref{eq:fftborapp:zx}),
(\ref{eq:fftborapp:zbx}), and
(\ref{eq:fftborapp:zmx})
follow by induction, as in the proof of Theorem \ref{thm:fftbor:recursions}.
