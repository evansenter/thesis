%!TEX root = ../main.tex

\chapter{FFTbor2D}
\label{ch:ffttwo}

\lhead{FFTbor2D}

\section{Introduction}
\label{sec:ffttwo:intro}

In this chapter, we present the \ffttwo algorithm and accompanying software.
\ffttwo, like \fftbor described in Chapter \ref{ch:fftbor}, is an algorithm
which computes the paramerized partition function for an input RNA sequence
\seq. \ffttwo computes the two-dimensional coarse energy landscape for \seq
given two compatible input secondary structures \strA and \strB, where position
$(x,y)$ on the discrete energy landscape corresponds to the Boltzmann
probability for those structures \str which have $d_{BP}(\str, \strA)=x$ and
$d_{BP}(\str, \strB)=y$ (where $d_{BP}$ is as defined in equation
\ref{eq:fftbor:dBP}). By again leveraging the \fft, \ffttwo runs in \On{5}
time and only uses \On{2} space---a significant improvement over previous
approaches. This permits the output energy landscape to be used in a
high-throughput fashion to analyze folding kinetics; a topic covered in detail
in Chapter \ref{ch:hermes}.

\subsection{Organization}
\label{subsec:ffttwo:org}

This chapter is organized in the following fashion. Because the history for
this work arises naturally from the background described in section
\ref{sec:fftbor:bkgrnd}, we forego reiteration and instead fall directly into
a technical discussion of the underlying algorithm. We first develop the
recursions for the Nussinov energy model for expository clarity, the
underlying implementation uses the more complicated and robust Turner energy
model. Recursions in place, we then move to show how these lead to
a single variable polynomial $P(x)$ whose coeffecients can be computed by
the \idft, and map to the 2D energy landscape. We describe two exploitations of
$P(x)$, a parity condition and complex conjugates which further reduce the
runtime by a factor of 4. Finally, we contrast this software against \rtwofold,
and outline the performance characteristics of both softwares and highlight
the benefits and drawbacks of both.

\section{Derivation of the \ffttwo algorithm}
\label{sec:ffttwo:math}

\subsection{Definition of the partition function
\texorpdfstring{\bfZ{x,y}{1,n}}{}}
\label{subsec:ffttwo:recursions}

\subsection{Recursions to compute the polynomial
\texorpdfstring{\emZ{i,j}}{}}
\label{subsec:ffttwo:polynomial}

\subsection{Polynomial interpolation to evaluate
\texorpdfstring{\emZ{i,j}}{}}
\label{subsec:ffttwo:fft}

\section{Acceleration of the \ffttwo algorithm}
\label{sec:ffttwo:perf}

\subsection{Optimization due to parity condition}
\label{subsec:ffttwo:parity}

\subsection{Optimization due to complex conjugates}
\label{subsec:ffttwo:compconj}

\section{Benchmarking and performance considerations}
\label{sec:ffttwo:benchmarking}

\section{Applications of the \ffttwo algorithm}
\label{sec:ffttwo:applications}
