%!TEX root = ../main.tex

\chapter{Introduction}
\label{ch:intro}

\lhead{Introduction}

Introduced in $1958$, the central dogma of biology has been an excellent model for
the biological flow of information, much as Newtonian classical mechanics stood the
test of time for over $200$ years. But just as Einstein’s revolutionary principle of
relativity has upended our understanding of space in a way unheard of since
Copernicus, recent research has gone to confirm that for all our scientific
progress, the cell still holds fundamental mysteries, and even the central dogma
isn't sacred. Even Francis Crick himself indicated in \citep{crick:1970wb} that
ribonucleic acids (RNAs) likely had a role beyond the traditional {\em messenger}
intermediary between
DNA and proteins, as evidenced by viral RNAs \citep{coffin:1997ws}. Though recent
research continues to suggest that the genome is pervasively transcribed, current
estimates indicate that only 1.2\% of the mammalian genome constitutes
protein-coding sequences \citep{berretta:2009tq,clark:2011cc,jensen:2013vb}. The
study of the transcriptome has led to the identification of a wide variety of
non-coding RNAs (ncRNAs) that highlight the diversity of roles for which RNA can
be put to use \citep{costa:2005ug}. Now understood to be much more than the
intermediary step between DNA and proteins, RNAs have been implicated in a variety
of regulatory and enzymatic activities, including gene knockdown and silencing
\citep{fire:1998tv,mccaffrey:2002tf,hannon:2002vn,he:2004uk},
transcriptional and translational regulation \citep{nudler:2004vm,mandal:2004vh},
intronic splicing \citep{kruger:1982wk,cech:1990tn}, cite-specific cleavage
\citep{doherty:2001wq}, and more.
A prevailing theory now suggests that self-replicating RNA molecules were the
predecessors to all life on Earth---the RNA world hypothesis \citep{gilbert:1986td}.
As our appreciation of RNA diversity has increased,
significant effort has been put forth by the scientific community to understand
and characterize the properties of these molecules.

Unlike DNA, RNA is generally single stranded and thus able to interact with itself
to form interesting shapes with various functional characteristics, akin to
proteins. RNA is a polymer comprised of four monomer {\em building blocks}:
the purines
adenine (A) and guanine (G); and the pyrimidines cytosine (C) and uracil (U). These
nucleotides can form planar base pairs comprised of energetically favorable
hydrogen bonds, the stacking of which produces a stable {\em helix} structure
\citep{yakovchuk:2006bm}. There are only a select set of possible base pairs; the
Watson-Crick pairs (A-U, G-C) or the G-U wobble pair. Given an arbitrary RNA
sequence $\seq = \seqN$, where $s_i \in \{\text{A,\,U,\,G,\,C}\}$, we can define
a secondary structure \str for \seq as the set of index tuples indicating those
bases involved in a base pair within \seq. Again like proteins, RNA molecules tend
to fold into a `native' conformation, usually that which minimizes free energy.
While protein folding is predominantly motivated by hydrophobic
interactions, RNA structure is driven by stacking base pair interactions, and
therefore secondary structure tends to be a much better predictor for the function
of the molecule in question than is the case with proteins, whose function is
largely determined by 3D `tertiary' structure.

From a computational perspective, the history of RNA folding is far too long for
proper treatment within this introduction. Instead we will touch upon just some
of the major milestones to get a flavor for what progress has been made. In $1978$,
Michael Waterman presented the first graph-theoretic model of single stranded
nucleic acids such as RNA \citep{waterman:1978va}. This was followed in $1980$ by
the work of Ruth Nussinov and Ann Jacobson, who together presented an algorithm for
determining the maximally matching secondary structure \str for
a given RNA sequence \citep{nussinov1980}, using dynamic programming
\citep{bellman:1952vza}. In the following years Michael Zuker and Patrick Stiegler
developed an algorithm and accompanying software for the \mfe
formulation of the problem \citep{zuker:1981tf,zuker:1989im}. In $1990$, John
McCaskill showed how dynamic programming could be used to compute the
partition function for an RNA molecule, and even compute the probability that an
arbitrary base is bound \citep{mccaskill}. Alongside these early developments,
more robust energy models were experimentally derived \citep{turner,turner:2009vy},
further improving the accuracy of computational models.

Fast-forwarding to today, there is now a huge collection of software aimed at
computing various properties of RNAs, be it folding, inverse folding, kinetics,
design, and more. The work that we present here intends to be a contribution
to the diverse toolset that researchers have at their disposal for the analysis
and design of both existing and novel RNA sequences.

\section{Thesis Content}
\label{sec:intro:thesiscontent}

The work of this thesis is based on the following four journal articles, alongside
unpublished data and observations. The journal articles constituting the
primary body of research include:

\begin{itemize}
\setstretch{1.3}
\item Senter, E., Sheikh, S., Dotu, I., Ponty, Y., \& Clote, P. (2012). Using the Fast Fourier Transform to Accelerate the Computational Search for RNA Conformational Switches. PloS One, 7(12), e50506. \url{http://doi.org/10.1371/journal.pone.0050506}
\item Ding, Y., Lorenz, W. A., Dotu, I., Senter, E., \& Clote, P. (2014). Computing the Probability of RNA Hairpin and Multiloop Formation. Journal of Computational Biology : a Journal of Computational Molecular Cell Biology, 21(3), 201–218. \url{http://doi.org/10.1089/cmb.2013.0148}
\item Senter, E., Dotu, I., \& Clote, P. (2014). RNA folding pathways and kinetics using 2D energy landscapes. Journal of Mathematical Biology, 70(1-2), 173–196. \url{http://doi.org/10.1007/s00285-014-0760-4}
\item Senter, E., \& Clote, P. (2015). Fast, Approximate Kinetics of RNA Folding. Journal of Computational Biology : a Journal of Computational Molecular Cell Biology, 22(2), 124–144. \url{http://doi.org/10.1089/cmb.2014.0193}
\end{itemize}

Text, figures, and tables from these papers are used throughout this thesis without
additional notice.

\section{Thesis Organization}
\label{sec:intro:thesisorg}

The remainder of this thesis is organized in the following fashion. We begin in
\Chref{rfinder} with the presentation of \rfinder, a pipeline of
software intended to facilitate the detection of full \rb sequences
alongside their corresponding `on' and `off' structures in genomic data. In
\Chref{fftbor} we introduce the program \fftbor, which computes---for
each integer $k$---the Boltzmann probability \pk of the subensemble of structures
whose \bpd to an input reference structure \str is $k$.
In \Chref{ffttwo} we extend this idea to simultaneously consider two
reference structures \strST and produce as a result the coarse-grained 2D energy
landscape where---for each integer pair $x,y$---we compute the Boltzmann
probability $p_{x,y}$
of those structures whose \bpd from \strS [resp. \strTresp] is $x$
[resp. $y$]. This program---\ffttwo---allows for the efficient approximation of
kinetic characteristics of RNA molecules, presented in \Chref{hermes}
through the software package \hermes. Finally in \Chref{disc} we conclude
with a summary of this work as a whole, and consider its place in the greater
ecosystem of computational RNA tools.

