%!TEX root = ../main.tex

\chapter{Hermes}
\label{ch:hermes}

\lhead{Hermes}

\section{Introduction}
\label{sec:hermes:intro}

In this chapter, we present the \hermes software suite---a collection of
programs aimed at evaluating the kinetic properties of RNA molecules.
Provided a coarse-grained energy landscape generated by \ffttwo (described
in Chapter \ref{ch:ffttwo}), we present software which computes both the \mfpt
and \eqt for this discretized energy landscape. We also provide software which
computes the exact kinetics for an RNA molecule, however since this requires
exhaustive enumeration of all secondary structures---which is known to be an
exponential quantity for the length of the RNA in consideration---the full
kinetics are not expected to be practical for anything beyond a sequence of
trivial length. The software in \hermes presents a practical application of
the energy landscapes computed by the \ffttwo algorithm. Contrasted against
the other kinetics software in the field, \hermes offers similar accuracy
with unparalleled performance which opens up the possibility for large-scale
kinetic analysis {\em in silico}, which we expect to be of use for synthetic
design.

\subsection{Organization}
\label{subsec:hermes:org}

This chapter is organized in the following fashion. We begin by providing
background on the state-of-the-art approaches for kinetic analysis of RNAs.
From there, we move into a technical discussion of two traditional approaches
for kinetics, computation of the \mfpt and the \eqt. With this foundation in
place, we proceed to discuss the high-level organization of the \hermes
software package, and describe in detail each of the four underlying programs
which comprise the kinetics suite. We then move on to present comparitive
benchmarking of \hermes against other methods, before finally concluding with
some remarks on the accuracy and applicability of \hermes to computational
RNA design.

\section{Background}
\label{sec:hermes:bkgrnd}

\section{Traditional approaches for kinetics}
\label{sec:hermes:kinetics}

\subsection{\Mfpt}
\label{subsec:hermes:mfpt}

\subsection{\Eqt}
\label{subsec:hermes:eq}

\section{Software within the \hermes suite}
\label{sec:hermes:layout}

The \hermes software package was developed on the Macintosh OS X
operating system (10.9.2 and 10.10) and should work with any Unix-like platform
(Ubuntu, Debian, and CentOS were tested). We make the source code freely
available under the MIT License in two locations.
Our lab hosts the latest stable version of the code at
\url{http://bioinformatics.bc.edu/clotelab/Hermes} and a fully
version-controlled copy at \url{https://github.com/evansenter/hermes}.
The data and figures presented in this article were generated with the
source code hosted at the first URL, and we make no guarantee as to
the stability of development branches in our Git repository.

External dependencies for the software include a C (resp. C++)
compiler supporting the GNU99 language specification (resp. C++98),
FFTW implementation of Fast Fourier Transform \cite{FFTW05} ($\geq
3.3.4$) \url{http://www.fftw.org/}, Gnu Scientific Library GSL ($\geq
1.15$) \url{http://www.gnu.org/software/gsl/}, Vienna RNA Package
\cite{Lorenz.amb11} ($\geq 2.0.7$) \url{http://www.viennarna.at}, and
any corresponding sub-packages included with the aforementioned
software. For a more detailed explanation of both external
dependencies and installation instructions, refer to the
`DOCS.pdf' file at the web site
outlining the configuration and compilation process for the \hermes
suite.

\hermes is organized into three independent directories: (1)
\ffttwo, (2) \rnamfpt, and (3) \rnaeq (see
Figure~\ref{fig:organizationHermes}). These packages compile into both
standalone executables and archive files. The archives provide a DRY
API which allow the development of novel applications using source
from across the \hermes package without having to copy-and-paste
relevant functions. We provide two such examples of this in the {\tt
ext} subdirectory: \fftmfpt and \ffteq. These applications
are simple C drivers that use functions from \ffttwo, \rnamfpt and
\rnaeq to replicate a pipeline of executable calls without having to
deal with intermediary data transformation, I/O between calls or
slow-down due to a scripting language driver such as Python or R.

\subsection{Exact \mfpt with \rnamfpt}
\label{subsec:hermes:rnamfpt}

\rnamfpt computes the {\em mean first passage time} (MFPT), sometimes
referred to as the {\em hitting time} of a Markov chain, by using
matrix inversion \cite{meyerMFPT} -- see
Section~\ref{section:algo}. The program takes as input a
comma separated value (CSV) file containing the non-zero positions and
values of a 2D probability grid; i.e. a CSV format file having columns {\em i},
{\em j}, and {\em p}. The first two columns, {\em i} and {\em j}
correspond to the {\em 0-indexed} row-ordered position in the rate
matrix and the final column {\em p} is the probability $p_{i,j}$
of a transition from {\em i} to {\em j}.  From this input, the
mean first passage time is constructed by matrix inversion.
Because this program was designed with the original intent
of handling 2D-probability grids, all vertexes are uniquely identified
by index tuples (which conceptually correspond to positioning in a 2D
array). However, it is trivial to use this code with both
1D-probability grids such as those produced by \fftbor
\cite{Senter.po12} or arbitrary transition matrices without any change
to the underlying implementation. The software additionally provides
many options for defining the structure of the graph underlying the
Markov chain. Some of these include the option to force a fully
connected graph (useful in cases where there is no non-zero path
between the start / end state) or to enforce detailed balance.
Finally, \rnamfpt also accepts as input the probability transition matrix,
a stochastic matrix with row sums equal to $1$, and computes the mean
first passage time for the corresponding Markov chain.

\subsection{Approximate \mfpt with \fftmfpt}
\label{subsec:hermes:fftmfpt}

\fftmfpt approximates the mean first passage time of a given RNA
sequence folding from input structure A to B, by {\em exactly}
computing the mean first passage time from state $(0,d_0)$ to state
$(d_0,0)$ in the 2D probability grid obtained from running
\ffttwo. Here, $d_0$ is the base pair distance between structures
$A,B$, and the MFPT is computed for the Markov chain, whose states are
the non-empty 2D probability grid points, and whose transition
probabilities are defined by $p_{(x,y),(x',y')} =
\frac{P(x',y')}{P(x,y)}$. As we report in this paper, given an RNA
sequence $\seq$, if $A$ is the empty structure and $B$ the MFE
structure of $\seq$, then \fftmfpt output is well correlated with the
exact MFPT in folding the empty structure to the MFE structure, where
transitions between structures involve the addition or removal of a
single base pair.

\subsection{Exact \eqt with \rnaeq}
\label{subsec:hermes:rnaeq}

\rnaeq computes the population proportion of a user-provided structure
over arbitrary time units. Like \rnamfpt, this program takes as input a
comma separated value (CSV) file containing the non-zero positions and
values of a 2D probability grid. From this input a rate matrix
is constructed for the underlying Markov process. Alternatively,
\rnaeq can accept as input an arbitrary rate matrix. Performing
spectral decomposition of the column-ordered rate matrix that
underlies the corresponding Markov process, \rnaeq computes either
the equilibrium time or population occupancy frequencies.
Additionally, \rnaeq can call the
Vienna RNA Package program \rnasub \cite{Wuchty.b99}, with a
user-specified upper bound to the energy difference with the minimum
free energy. With this option, the rate matrix is
constructed for the Markov process, whose states consist of all
the structures returned by \rnaeq, and the equilibrium time or
population occupancy frequencies are computed. Due to the time
and memory required for this option, we do not expect it to be used except for
small sequences.

\subsection{Approximate \eqt with \ffteq}
\label{subsec:hermes:ffteq}

\ffteq allows an investigator to efficiently estimate population
kinetics for a sequence folding between two arbitrary, but fixed,
structures. The transition rate matrix underlying the Markov process
necessary for eigendecomposition is derived from the 2D-energy
landscape. Vertices in the rate matrix represent a subset of
structures compatible with the input sequence as modeled by
\ffttwo, which makes the graph size more tractable than structural
sampling with {\tt RNAsubopt}, even with constraints.

\subsubsection{Population occupancy curves with \ffteq}
\label{subsubsec:hermes:popcurve}

\subsubsection{Approximating equilibrium time from occupancy curves}
\label{subsubsec:hermes:eqestimate}

The computation of an equilibrium time value from the eigendecomposition of the rate matrixis a rather thorny issue. (something about the non-linear solver approach not working, same as in FFTbor w/ numeric instability)

\section{Correlations of kinetics data across software}
\label{sec:hermes:corr}

\subsubsection{Benchmarking data for computational comparison}
\label{subsubsec:hermes:benchdata}

In this section, we describe a
benchmarking set of 1,000 small RNAs used to benchmark the previously
described kinetics methods in a comparative study. To ensure that mean
first passage time can be computed from
$(I - P^{-}_{x_{\infty}})^{-1} \cdot {\bf e}$ by using matrix
inversion, that spectral decomposition of the rate matrix is possible,
and to ensure that \kinfold simulations would provide sufficient
statistics, we generated a collection of 1,000 random RNA sequences of
length 20 nt, each having expected compositional frequency of $1/4$
for A,C,G,U, and each having at most 2,500 distinct secondary
structures, such that the minimum free energy is less than or equal to
$-5.5$ kcal/mol.

For example, one of the 1,000 sequences is ACGCGACGUGCACCGCACGU with
minimum free energy structure {\tt .....((((((...))))))} having free
energy of $-6.4$ kcal/mol. Statistics for the free energies of the
2,453 secondary structures of this 20-mer are the following: mean is
$10.695$, standary deviation is $4.804$, maximum is $25.00$, minimum
is $-6.40$. A histogram for the free energy of all secondary
structures of ACGCGACGUGCACCGCACGU is depicted in
%Figure~\ref{fig:freeEnergyHistogramSecStrACGCGACGUGCACCGCACGU}.
the left panel of Figure~\ref{fig:PLMVd}. The right panel of the
same figure depicts the minimum free energy structure of the
54 nt hammerhead type III ribozyme from Peach Latent Mosaic Viroid
(PLMVd), discussed later. This secondary structure is identical
to the consensus structure from Rfam 11.0 \cite{Gardner.nar11}.


\begin{figure*} [!h]
\begin{center}
% \includegraphics[width=0.45\linewidth]{figure2a.eps}
% \includegraphics[width=0.45\linewidth]{figure2b.eps}
\caption{\small
{\em (Left)} Histogram of free energies of secondary structures of
ACGCGACGUGCACCGCACGU, which range from $-6.5$ to $+25$ kcal/mol, with
mean of $10.695$ kcal/mol.
{\em (Right)} Minimum free energy structure of the 54 nt Peach Latent Mosaic
Viroid (PLMVd) AJ005312.1/282-335, which is identical to the consensus
structure from Rfam 11.0 \cite{Gardner.nar11}. {\tt RNAfold} from
Vienna RNA Package 2.1.7 with energy parameters from the Turner 1999
model were used, since the minimum free energy structure determined by
the more recent Turner 2004 energy parameters
does {\em not} agree with the Rfam consensus structure -- see
\cite{syntheticHammerheads}. Positional entropy, a measure
of divergence in the base pairing status at each positions for the
low energy ensemble of structures, is indicated by color, using the
RNA Vienna Package utility script {\tt relplot.pl}.
}
\label{fig:PLMVd}
\end{center}
\end{figure*}

Figure~\ref{fig:meanStdevKinfoldRuns1000sequences} displays the mean
and standard deviation for \kinfold simulations of folding time
for each of the 1,000 RNA sequences from our benchmarking data. For
each sequence, the mean and standard deviation of the time required to
fold the empty structure to the MFE structure were computed from
10,000 \kinfold runs, each run with an upper bound of $10^8$
Monte Carlo steps, thus ensuring that all simulations converged. The
sequences were then sorted by increasing folding time mean. Standard
deviation exceeded the mean in $83.9\%$ of the 1,000 cases, indicating
the enormous variation between separate \kinfold runs, even for
20 nt RNA sequences having at most 2,500 secondary structures. In our
opinion, \kinfold is an expertly crafted implementation of
Gillespie's algorithm for an event driven Monte Carlo simulation of
one-step RNA secondary structure folding. From the standpoint of
biophysics and physical chemistry, there is no more reliable
simulation method, except of course the exact computation of mean
first passage time using linear algebra. Nevertheless, the enormous
time required for reliable \kinfold estimations and the large
standard deviations observed point out the need for a faster method to
approximate folding time.

\begin{figure*}
\centering
% \includegraphics[width=0.45\textwidth]{figure3a.eps}
% \includegraphics[width=0.45\textwidth]{figure3b.eps}
\caption{{\em (Left)} Histogram of \kinfold folding times for 20-mer CCGAUUGGCG AAAGGCCACC. The mean [resp. standard deviation] of 10,000 runs of \kinfold for this 20-mer is 538.37 [resp. 755.65]. Note the close fit to the exponential distribution, {\em (Right)} Mean minus standard deviation ($\mu -\sigma$), mean ($\mu$), and mean plus standard deviation ($\mu + \sigma$) of the logarithm of \kinfold folding times, taken over 10,000 runs for each of the 1,000 sequences from the benchmarking set of 20-mers. For graphical illustration, we have sorted the log folding times in increasing order. } \label{fig:meanStdevKinfoldRuns1000sequences}
\end{figure*}

\subsubsection{Pearson correlation coefficients for various kinetics packages}
\label{subsubsec:hermes:corrtable}

In this section, we display the correlation between (1) the {\em gold
standard} method {\tt MFPT}, both with and without the Hastings
modification using equations (\ref{eq:MFPTwithHastings}) and
(\ref{eq:MFPTwithoutHastings}), (2) the {\em platinum standard} method {\tt
Equilibrium}, (3) the {\em silver standard} method \kinfold, (4) {\tt
FFTmfpt} with and without the Hastings modification using equations
(\ref{eq:transitionProbabilityFFTbor2DwithHastings}) and
(\ref{eq:transitionProbabilityFFTbor2DwithoutHastings}), (5) {\tt
FFTeq} which computes equilibrium time for the 2D-grid, (6) {\tt
RNA2Dfold } with and without the Hastings modification using equations
(\ref{eq:transitionProbabilityFFTbor2DwithHastings}) and
(\ref{eq:transitionProbabilityFFTbor2DwithoutHastings}). Correlations
with [resp. without] the Hastings modification are summarized in the
lower [resp. upper] triangular portion of
Table~\ref{table:correlation}. It is clear that correlations between
the mathematically exact methods {\tt MFPT}, {\tt Equilibrium}, and
approximation methods \kinfold, {\tt FFTmfpt}, {\tt FFTeq}, {\tt
RNA2Dfold} are improved when using the Hastings correction.

Figures~\ref{fig:scatterplot_single_bp_moves_for_kinfold_vs_rnaeq},
\ref{fig:scatterplot_single_bp_moves_for_actual_mfpt_vs_fftbor2d},
\ref{fig:scatterplot_single_bp_moves_for_fftbor2d_vs_rnapopulation}
depict scatterplots for kinetics obtained by some of the algorithms
above. The left panel of
Figure~\ref{fig:scatterplot_single_bp_moves_for_kinfold_vs_rnaeq}
shows a scatter plots for gold standard {\tt MFPT} versus platinum
standard {\tt Equilibrium}, with correlation value 0.5652. The right
panel of the same figure shows a scatter plot for \kinfold versus
{\tt Equilibrium}, with correlation 0.7814. Note the persence of two
clusters in this and some of the other scatter plots. Cluster A
consists of RNA sequences whose folding time, as determined by {\tt
MFPT} or {\tt Equilibrium}, is rapid -- specifically, the natural
logarithm of the MFPT is at most 7.5. Cluster B consists of the
remaining RNA sequences, whose folding time is longer than that of
cluster A. There are no significant differences between RNA sequences
in clusters A and B with respect to GC-content, sequence logo, minimum
free energy, number of secondary structures, etc.  The left panel of
Figure~\ref{fig:scatterplot_single_bp_moves_for_actual_mfpt_vs_fftbor2d}
shows the scatter plot for {\tt MFPT} versus \kinfold, with
correlation 0.7933, and the right panel shows the scatter plot for
{\tt MFPT} versus {\tt FFTmfpt}, with correlation 0.6035.
Figure~\ref{fig:scatterplot_single_bp_moves_for_fftbor2d_vs_rnapopulation}
shows scatter plots for {\tt FFTmfpt} versus \kinfold (left) and
for {\tt FFTmfpt} versus {\tt FFTeq} (right), with respective
correlation values 0.7608 and 0.9589. \kinfold obviously provides
a better correlation with the exact value of mean first passage time;
however, since the standard deviation of \kinfold runs is as
large as the mean,\footnote{It follows from spectral decomposition that
equilibrium time follows an exponential distribution (or sum of
exponential distributions). Exponential distributions have the property
that the mean is equal to the standard deviation, hence
it is not surprising that \kinfold
kinetics have this property.} accurate kinetics estimates
from \kinfold require prohibitively large computational time -- indeed, in
\cite{wolfingerStadler:kinetics} reliable kinetics for phe-tRNA from
yeast were obtained by 9,000 \kinfold simulations, each for $10^8$
steps, requiring 3 months of CPU time on an Intel Pentium 4 running at
2.4 GHz under Linux. Although the correlation value of 0.6035 between
{\tt MFPT} and {\tt FFTmfpt} is much less than that obtained by {\tt
Kinfold}, the runtime required by our method {\tt FFTmfpt} is measured
in seconds, even for moderate to large RNAs. For this reason, we
advocate the use of {\tt FFTmfpt} in synthetic biology screens to
design RNA sequences having certain desired kinetic properties. Once
promising candidates are found, it is possible to devote additional
computational time to \kinfold simulations for more accurate
kinetics.


\begin{figure}
\centering
% \includegraphics[width=0.45 \textwidth]{figure4a.eps}
% \includegraphics[width=0.45 \textwidth]{figure4b.eps}
\caption{ Scatter plots of the natural logarithm of times from {\tt
MFPT} versus {\tt Equilibrium} (left) and for \kinfold versus
{\tt Equilibrium} (right). }
\label{fig:scatterplot_single_bp_moves_for_kinfold_vs_rnaeq}
\end{figure}


\begin{figure}
\centering
% \includegraphics[width=0.45 \textwidth]{figure5a.eps}
% \includegraphics[width=0.45 \textwidth]{figure5b.eps}
\caption{ Scatter plots of the natural logarithm of times from {\tt
MFPT} versus \kinfold (left) and for {\tt MFPT} versus {\tt
FFTmfpt} (right). }
\label{fig:scatterplot_single_bp_moves_for_actual_mfpt_vs_fftbor2d}
\end{figure}


\begin{figure}
\centering
% \includegraphics[width=0.45 \textwidth]{figure6a.eps}
% \includegraphics[width=0.45 \textwidth]{figure6b.eps}
\caption{ Scatter plots of the natural logarithm of times from {\tt
Kinfold} versus {\tt FFTmfpt} (left) and for {\tt FFTmfpt} versus {\tt
FFTeq} (right). }
\label{fig:scatterplot_single_bp_moves_for_fftbor2d_vs_rnapopulation}
\end{figure}

\begin{table}
\label{table:correlation}
\begin{tabular}
  {|l|llllllll|} \hline \small{Hastings (Yes\textbackslash No)} & \small{MFPT} & \small{Equilibrium} & \small{Kinfold} & \small{FFTmfpt} & \small{RNA2Dfold} & \small{FFTbor} & \small{BarrierBasins} & \small{FFTeq} \\
  \hline \small{MFPT}   & 1      & 0.5683 & 0.7945 & 0.5060 & 0.5110 & 0.5204 & 0.5280 & 0.4472 \\
  \small{Equilibrium}   & 0.5798 & 1      & 0.7814 & 0.7043 & 0.7025 & 0.5080 & 0.5979 & 0.6820 \\
  \small{Kinfold}       & 0.7933 & 0.7507 & 1      & 0.7312 & 0.7358 & 0.6241 & 0.6328 & 0.6445 \\
  \small{FFTmfpt}       & 0.6035 & 0.7935 & 0.7608 & 1      & 0.9980 & 0.5485 & 0.8614 & 0.9589 \\
  \small{RNA2Dfold}     & 0.6076 & 0.7919 & 0.7655 & 0.9983 & 1      & 0.5584 & 0.8538 & 0.9515 \\
  \small{FFTbor}        & 0.5416 & 0.5218 & 0.6241 & 0.5748 & 0.5855 & 1      & 0.3450 & 0.4229 \\
  \small{BarrierBasins} & 0.6346 & 0.6578 & 0.6328 & 0.8310 & 0.8217 & 0.3450 & 1      & 0.9149 \\
  \small{FFTeq}         & 0.5614 & 0.7916 & 0.6966 & 0.9670 & 0.9590 & 0.4757 & 0.8940 & 1      \\
  \hline
 \end{tabular}

 \caption{Table of Pearson correlation coefficients for various methods to compute or approximate RNA secondary structure folding kinetics. Lower [resp. upper] triangular entries are with [resp. without] the Hastings correction for Markov chain probability matrices. The methods are: {\tt MFPT} (mean first passage time, computed by matrix inversion for the Markov chain consisting of all secondary structures, with move allowed between structures differing by one base pair), {\tt Equilibrium} (equilibrium time, computed by spectral decomposition of a rate matrix comprising all secondary structures to compute population fraction $P(t)$ at time $t$), \kinfold (an implementation of Gillespie's Algorithm to approximate refolding pathways using an event-based Monte Carlo simulation), {\tt FFTmfpt} (mean first passage time for Markov chain consisting of ``grid point'' states $(x,y)$ with probability $P(x,y)=\sum_S exp(-E(S)/RT)/Z$, computed by {\tt FFTbor2D}, where the sum is taken over structures having base pair distance $x$ to the empty structure and $y$ to the MFE structure, {\tt RNA2Dfold} (mean first passage time, computed as previously explained, but using {\tt RNA2Dfold} in place of {\tt FFTbor2D} to compute $P(x,y)$), \fftbor (mean first passage time, computed for the Markov chain consisting of states $0,1,\ldots,n$, for which $P(x) = \sum_S \exp(-E(S)/RT)/Z$, where the sum is taken over all secondary structures whose base pair distance is $x$ from the MFE structure), {\tt BarrierBasins} (equilibrium time, computed using spectral decomposition on the Markov process consisting of ``grid point'' states output from {\tt Barriers}), and {\tt FFTeq} (equilibrium time, computed in the same fashion as {\tt BarrierBasins} using a Markov process derived from the energy landscape output by {\tt FFTbor2D}). }
 \end{table}
