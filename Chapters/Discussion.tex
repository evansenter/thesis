%!TEX root = ../main.tex

\chapter{Discussion}
\label{ch:disc}

\lhead{Discussion}

Over the course of this thesis, we have described a collection of tools aimed
at facilitating the computational analysis of RNA. The work in \Chref{rfinder}
aims to address the open problem of determining full \rb sequences, along
with their `on' and `off' conformational states in an efficient and generalized
manner. \Rbs are an imporant regulatory motif that can influence
transcription by the introduction of an intrinsic terminator, an extended stem-loop
followed by a series of uracil residues which cause RNA polymerase to prematurely
terminate transcription \citep{gusarov:1999uu,yarnell:1999wt}. Alternately,
\rbs can regulate translation, generally by sequestration of the ribosome
binding site in response to the presence or absence of a corresponding ligand
\citep{barrick:2007gw}. This degree of control makes \rbs well suited for
the regulation of metabolic pathways, and are generally found within bacterial
genomes. While the aptamer portion of the \rb necessarily exhibits high
sequence and structural conservation---necessary to retain a high binding affinity
for the corresponding ligand---the downstream expression platform is much more
variable, and difficult to detect using covariance models such as \infernal
\citep{infernal}. For this reason, databases such as Rfam \citep{nawrocki:2014uy}
only contain alignments for the conserved aptamer domain; rarely enough to
understand the structural characteristics of the \rb `states' in the greater
context of the mRNA. \rfinder aims to provide a holistic description of
\rbs by coupling powerful aptamer-detection tools (such as \infernal) with software
designed to detect intrinsic terminators (such as \tthp), and use
experimentally-derived structural information from the literature in conjunction
with constrained folding methods to predict full gene `on' and `off' conformations
in a general manner.

The remainder of the thesis follows a natural progression. In \Chref{fftbor} we
present the algorithm \fftbor, an efficient approach to compute the
{\em parameterized} Boltzmann probabilities $\pk = \frac{\bfZ{k}{}}{\fullZ}$ of
those sequences having \bpd $k$ from an arbitrary input structure
\strSt. By using the \fft along with \nRoUs, \fftbor operates in \On{4} speed,
an order of magnitude faster than its predecessor. This approach of applying the
\fft to compute parameterized partition functions was put to use in
\citep{ding:2014ex} as well, where we develop programs \ms{FFThairpin}
[resp. \msresp{FFTmultiloop}] to describe the probability distribution of structures
having exactly $k$ hairpins [resp. multiloops]. Though not discussed here, by
training a support vector machine (SVM) on the probability distributions output
by these
programs we were able to classify Rfam families using five-fold cross validation
with excellent `area under curve' (AUC) values.

\Chref{ffttwo} uses the \fft to compute the 2D probability distribution, where
given two compatible input secondary structures \strA and \strB, position
$(x,y)$ on the discrete energy landscape corresponds to the Boltzmann
probability for those structures \str which have \bpd $x$ [resp. $y$]
from \strA [resp. \strBresp]. Compared against its closest relative \rnatwofold,
\ffttwo is two orders of magnitude faster, operating in \On{5} speed and only
\On{2} space. This opens the door for using \ffttwo in a high-throughput fashion,
where the probability landscapes can be used as a starting point for kinetic
analysis.

With \hermes in \Chref{hermes}, we do just that; representing the output of
\ffttwo as a graph, we can create a corresponding Markov chain [resp. Markov
process] to estimate \mfpt [resp. equilibrium time] with correlation to the
mathematically-exact values as shown in Table \ref{table:correlationHermes}.
The speed of these approximate tools within \hermes---namely \fftmfpt and
\ffteq---present what we believe to be the first suite of kinetic analysis
tools for RNA sequences that are suitable for high throughput usage, something
we believe to be of interest in the field of synthetic design.

Finally, though not discussed in the body of this thesis, we have also
developed a collection of packages for working with command-line tools for
computational RNA analysis in a more streamlined fashion. Two of these programs,
\href{https://github.com/evansenter/wrnap}{\wrnap} and
\href{https://github.com/evansenter/rbfam}{\rbfam} are among the first of their
type for the Ruby programming langage, a popular interpreted language that
enjoys wide adoption in the web development community. In an era of increasingly
cloud-focused software, we believe that researchers will need a robust set of
tools for working with their data in a web-oriented environment, something
\wrnap and \rbfam aim to provide.
