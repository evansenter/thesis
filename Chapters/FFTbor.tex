%!TEX root = ../main.tex

\chapter{FFTbor}
\label{ch:fftbor}

\lhead{FFTbor}

\section{Introduction}
\label{sec:fftbor:intro}

In this chapter, we present the \fftbor algorithm and accompanying software.
\fftbor is a novel algorithm developed with the intent of efficiently computing
the Boltzmann probability of those structures whom, for a given input RNA
sequence \seq, differ by $k$ base pairs. By leveraging polynomial interpolation
via the \fft, this algorithm runs in \On{4} time and
\On{2} space, a significant improvement over its predecessor. The accompanying
software which implements this algorithm has been used to predict the location
of expression platforms for putative \rbs in genomic data, and to
evaluate the correlation between kinetic folding speed and landscape ruggedness.

\subsection{Organization}
\label{subsec:fftbor:org}

This chapter is organized in the following fashion. First, we provide
background on
the problem which \fftbor aims to address, as well as a brief overview of
existing approaches. We follow by a formal explanation of the problem, and
proceed to describe how the energy landscape is coarsified into discrete bins.
We then develop the recursions for the parameterized partition function using
the Nussinov-Jacobson energy model, which allows us to exposé the novel aspects
of the algorithm. After developing the recursions, we indicate how they can be
reformulated as a polynomial whose coefficients $c_k=\bfZ{k}{1,n}$. We then
describe how the \fft can be employed to efficiently compute the coefficients
$c_k$, finishing our description of the underlying algorithm. Then we proceed
to present two applications of \fftbor, an application to RNA kinetics and
another to \rb detection in genomic data. Finally, we give reference
to the full recursions using the more accurate Turner energy model, which the
underlying implementation actually uses.

\section{Background}
\label{sec:fftbor:bkgrnd}

\section{Formalization of the problem}
\label{sec:fftbor:formal}

\fftbor aims to compute the coefficients $p_0,\ldots,p_{n-1}$ of the polynomial
\begin{align}
\label{eq:fftbor:pOfX}
p(x) = p_0 + p_1 x + p_2 x^2 + \cdots + p_{n-1} x^{n-1},
\end{align}
where $p_k$ is defined as $p_k = \frac{Z_k}{Z}$. We employ the \fft to compute
the \idft on values $y_0,\ldots,y_{n-1}$, where
$y_k = p(\omega^k)$ and $\omega = e^{2 \pi i/n}$ is the principal $n$th complex
root of unity and $p(x)$ is defined in (\ref{eq:fftbor:pOfX}). By leveraging
complex
roots of unity in conjunction with the \idft the we subvert numeric instability
issues observed with both Lagrange interpolation and Gaussian elimination.

Consider an RNA sequence $\seq = s_1,\ldots,s_n$, where
$s_i \in \{A,C,G,U\}$, i.e. a sequence of nucleotides. We can describe a
secondary structure \str which is compatible with \seq as a collection of
base pair tuples $(i,j)$, where $1 \le i \le i+\theta < j \le n$ and
$\theta \ge 0$---generally taken to be 3, the minimum number of unpaired bases
in a hairpin loop due to steric constraints.

To more simply develop the underlying recursions for \fftbor, we introduce a
number of constraints on the base pairs within \str. Firstly, we require that
each base pair is either a Watson-Crick or G-U wobble, i.e. base pair $(i,j)$
for sequence \seq has corresponding nucleotides $(s_i,s_j)$, which are
restricted to the set
\[
\mathbb{B} = \{(A,U), (U,A), (G,C), (C,G), (G,U), (U,G)\}.
\]
With
this constraint satisfied we say that \str is {\em compatible} with \seq, and
for the remainder of this chapter will only consider those structures which are
compatible with \seq.
Secondly, we insist that given two base pairs $(i,j), (x,y)$ from \str,
$i=x \iff j=y$---bases have at most one partner. Finally, we require that
$i<x<j \iff i<y<j$, no pseudoknots are allowed. While pseudoknots have been
shown to be present in some biologically relevant RNAs, their inclusion greatly
complicates the recursive decomposition of the structure, and thus it is common
to ignore them.

Provided two secondary structures $\str,\strT$, we can define a notion of
distance between them. There are a number of different definitions of distance
used across the literature; we will use {\em \bpd} for \fftbor.
Base pair distance is defined as the symmetric difference between the sets
\str,\strT
\begin{align}
\label{eq:fftbor:dBP}
\dBP{\str,\strT} = |\str \cup \strT| - |\str \cap \strT|.
\end{align}
Given this definition of distance, two structures \str and \strT are said to
be \kNbrs if $\dBP{\str,\strT} = k$. It is important to note that
the notion of \bpd is also applicable to restrictions of secondary structures
on the subsequence $\seq_{i,j}$,
i.e. $\str_{i,j} = \{ (x,y) \,:\, i \leq  x < y \leq j,  (x,y) \in \str \}$.

For a restriction of base pairs for a given structure $\str_{i,j}$,
$\strT_{i,j}$ is said to be a \kNbr of $\str_{i,j}$ if
\[
\dBP{\str_{i,j},\strT_{i,j}} = |\{ (x,y): i \leq x<y\leq j,
(x,y) \in \str - \strT \mbox{ or } (x,y) \in \strT - \str \}| = k.
\]

\section{Derivation of the \fftbor algorithm}
\label{sec:fftbor:math}

Given an RNA sequence $\seq=s_1,\ldots,s_n$ and compatible secondary structure
\strSt, let \bfZ{k}{} denote the sum of the Boltzmann factors
\boltzf{\str} of all $k$-neighbors \str of \strSt; i.e.
\[
\bfZ{k}{} = \bfZ{k}{1,n} =
\sum_{\substack{\str \text{ such that }\\ \dBP{\str,\strSt}=k}}
\boltzF{\str}
\]
where $E(\str)$ denotes the Turner (nearest neighbor)
energy \cite{}
of \str, $R = 0.00198$ kcal/mol denotes the universal
gas constant and $T$ denotes absolute temperature. From this, it follows that
the full partition function is defined as
\begin{align}
\label{eq:fftbor:partFunc}
\bfZ{}{} = \bfZ{}{1,n} = \sum_{k=0}^n \bfZ{k}{1,n}
\end{align}
since the \bpd between \strSt and \str is at most
\begin{align}
\label{eq:fftbor:maxDist}
\dBP{\strSt,\str} \leq |\strSt| + \lfloor \frac{n-\theta}{2} \rfloor \leq n.
\end{align}

We can then define the Boltzmann probability of all \kNbrs \. of \strSt as
\begin{align}
\label{eq:fftbor:probK}
p(k) =\frac{\bfZ{k}{1,n}}{\bfZ{}{1,n}}.
\end{align}

By visualizing the probabilities $p_k$ as a function of $k$, we generate a
coarse grained view of the one-dimensional energy landscape of \seq with
respect to \strSt. When \strSt is taken to be the \mfes for example, one would
anticipate to see a peak at $k=0$, with additional peaks implying additional
metastable structures; local energy minima which could suggest an energetic
trap while folding.

\subsection{Definition of the partition function
\texorpdfstring{\bfZ{k}{1,n}}{}}
\label{subsec:fftbor:recursions}

For the rest of the paper, we consider both \seq as well as the
secondary structure \strSt on \seq to be fixed. We now recall the
recursions from Freyhult et al. \cite{Freyhult.ab05} to determine
the partition function \bfZ{k}{i,j} with
respect to the Nussinov-Jacobson
energy $E_0$ model \cite{nussinovJacobson}, defined by
$-1$ times the number of base pairs; i.e. $E_0(S) = -1 \cdot |S|$.
Although we describe here the recursions for the Nussinov-Jacobson
model, for the sake of
simplicity of exposition, both \rnabor
\cite{Freyhult.ab05} as well as our current software \fftbor,
concern the Turner energy model, consisting of free energy parameters for
stacked bases, hairpins, bulges, internal loops and multiloops. The full
recursions for \fftbor are described for the
the Turner energy model in the appendix.

The base case for \bfZ{k}{i,j} is given by
\begin{equation}
  \bfZ{0}{i}{j} = 1,\, \mbox{ for $i \le j$,}
  \label{eq:Zinit1}
\end{equation}
since the only $0$-neighbor to a structure \strSt
is the structure \strSt itself, and
\begin{equation}
  \bfZ{k}{i}{j} = 0,\, \mbox{for $k > 0$, $i \le j \leq
    i+\theta$},
  \label{eq:Zinit2}
\end{equation}
since the empty structure is the only possible structure for a
sequence shorter than $\theta + 2$ nucleotides, and so there are no
$k$-neighbors for $k>0$. The recursion used to compute
\bfZ{k}{i,j} for $k > 0$ and $j > i+\theta$ is
\begin{equation}
  \bfZ{k}{i}{j} = \bfZ{k -b_0}{i}{j-1} +
  \hspace{-5pt} \sum_{\substack{ (s_r,s_j) \in \mathbb{B}, \\
  i \leq r < j }} \sum_{w+w'=k - b(r)} \hspace{-10pt}
  \exp(-E_0(r,j)/RT) \cdot \bfZ{w}{i}{r-1} \bfZ{w'}{r+1}{j-1},
  \label{eq:N}
\end{equation}
where $E_0(r,j) = -1$ if positions $r,j$ can pair in sequence \seq,
and otherwise $E_0(r,j) = +\infty$. Additionally,
$b_0 = 1$ if $j$ is base-paired
in $\strSt_{[i,j]}$ and $0$ otherwise, and
$b(r)=d_{BP}(\strSt_{[i,j]}, \strSt_{[i,r-1]} \cup \strSt_{[r+1,j-1]} \cup\{(r,j)\})$.
This holds since in a secondary
structure $T_{[i,j]}$ on $s_i,\ldots,s_j$ that is a $k$-neighbor of
$\strSt_{[i,j]}$,
either nucleotide $j$ is unpaired in $[i,j]$ or it is
paired to a nucleotide $r$ such that $i \leq r < j$. In this
latter case it is enough to study the smaller sequence segments
$[i,r-1]$ and $[r+1,j-1]$ noting that, except for $(r,j)$,
base pairs outside of these regions are not allowed, since there
are no pseudoknots. In addition,
for $d_{BP}(\strSt_{[i,j]},T_{[i,j]}) = k$ to hold,
it is necessary for $w+w' = k -b(r)$ to hold, where $w =
d_{BP}(\strSt_{[i,r-1]},T_{[i,r-1]})$ and $w' =
d_{BP}(\strSt_{[r+1,j-1]},T_{[r+1,j-1]})$, since $b(r)$ is the
number of base pairs that differ between $\strSt_{[i,j]}$ and a
structure $T_{[i,j]}$, due to the introduction of the base pair
$(r,j)$.

\section*{Methods}

Given RNA sequence \seq and compatible initial structure \strSt,
we define the {\em polynomial}
\begin{eqnarray}
\label{eqn:polynomialYann}
\emZ{} = \sum_{k=0}^n c_k x^k
\end{eqnarray}
where coefficients $c_k=\bfZ{k}{1,n}$. Moreover, because of
(\ref{eqn:upperBound}) and the fact that the minimum number of
unpaired bases in a hairpin loop $\theta$ is $3$, we know that $c_n=0$,
so that \emZ{} is a polynomial of degree strictly less than $n$.
If we evaluate the polynomial \emZ{} for $n$ distinct values
\begin{eqnarray}
\label{eqn:ZofX}
\emZof{}{a_1} = y_1,\ldots, \emZof{}{a_n} = y_n,
\end{eqnarray}
then the Lagrange polynomial interpolation formula guarantees that
$\emZ{} = \sum_{k=1}^n y_k P_k(x)$, where the polynomials $P_k(x)$ have degree
at most $n-1$ and are given by the Lagrange formula
\begin{eqnarray}
\label{eqn:LagrangeInterpolation}
P_k(x) = \frac{\prod_{i\ne k} (x-x_i)}{\prod_{i \ne k} (x_k-x_i)}.
\end{eqnarray}
Since the polynomials $P_k(x)$ can be explicitly computed, it follows that
we can compute the coefficients $c_k$ of polynomial \emZ{}. As we describe
below, the evaluation of \emZ{} for a fixed value of $x$ can be done in
time $O(n^3)$ and space $O(n^2)$.  It follows that the coefficients
$c_k=\bfZ{k}{1,n}$ can be computed after
$n$ evaluations of \emZ{}, where the space for each evaluation of \emZ{}
is re-used; hence these evaluations can be performed in time $O(n^4)$ and space
$O(n^2)$. Finally,
Lagrange interpolation is clearly computable in time $O(n^3)$.
Although this approach is theoretically sound, there are severe
numerical stability issues related to the interpolation method
\cite{HighamBarycentricInterpolation},
the choice of values $a_1,\ldots,a_{n}$ in the interpolation,
and floating point arithmetic (round-off error) related to the
astronomically large values of the partition functions
\bfZ{k}{1,n}, for $0 \leq k < n$. After many unsuccessful
approaches including scaling (see Supplementary Information),
we obtained excellent results by
interpolating the polynomial $p(x)$, defined in equation~(\ref{eqn:PofX}),
rather than the polynomial \emZ{}, defined in equation~(\ref{eqn:ZofX}),
and performing interpolation with the Fast Fourier Transform (FFT) \cite{cormen}
where $\alpha_0,\ldots,\alpha_{n-1}$ are
chosen to be $n$th complex roots of unity,
$\alpha_k = e^{\frac{2 \pi k}{n}}$.
One
advantage of the FFT is that interpolation can be performed in $O(n \log n)$
time, rather than the cubic time required by using the Lagrange formula
(\ref{eqn:LagrangeInterpolation}) or by Gaussian elimination. Fewer
numerical operations implies increased numerical stability in our application.
Details now follow.

\subsection*{Recursions to compute the polynomial $\emZ{i,j}$}
\label{section:recursionsForPolynomialZij}

Given an initial secondary structure \strSt of a
given RNA sequence \seq, our goal is to compute
\begin{align}
\label{eqn:defZk}
\bfZ{k}{1,n} &= \sum_{\substack{S \text{ such that }\\ d_{BP}(S,\strSt)=k}}
e^{\frac{-E_0(S)}{RT}}
\end{align}
where $S$ can be any structure compatible with \seq.
As previously mentioned, the recurrence relation for \rnabor
with respect to the Nussinov energy model $E_0$ is
\begin{align}
\label{eqn:RNAborNussRecursion}
\bfZ{k}{i}{j} &= \bfZ{k-b_0}{i}{j-1} + \sum_{\substack{s_r s_j \in \mathbb{B},\\i\le r<j}}\left(e^{\frac{-E_0(r,j)}{RT}}\sum_{w+w'=k-b(r)}
\bfZ{w}{i}{r-1}  \bfZ{w'}{r+1}{j-1} \right)
\end{align}
where $E_0(r,j)=-1$ if $r$ and $j$ can base-pair and otherwise
$+\infty$, and
$b_0 = 1$ if $j$ is base paired in $\strSt_{[i,j]}$ and $0$ otherwise, and
$b(r)=d_{BP}(\strSt_{[i,j]}, \strSt_{[i,r-1]} \cup \strSt_{[r+1,j-1]} \cup\{(r,j)\})$.
The following theorem shows that an analogous recursion can be used to compute
the {\em polynomial} $\emZ{i,j}$ defined by
\begin{eqnarray}
\label{eqn:defZijpolynomial}
\emZ{i,j} = \sum_{k=0}^n c_k(i,j) x^k
\end{eqnarray}
where
\[ c_k(i,j)= \bfZ{k}{i}{j} =
\sum_{\substack{S \text{ such that }\\ d_{BP}(S,\strSt_{[i,j]})=k}}
e^{\frac{-E_0(S)}{RT}}.
\]
Here, in the summation, $S$ runs over structures on $s_i,\ldots,s_j$, which
are $k$-neighbors of the restriction $\strSt_{[i,j]}$ of initial structure
\strSt to interval $[i,j]$, and
$E_0(S)=-1 \cdot |S|$ denotes the Nussinov-Jacobson energy of $S$.
\medskip

\noindent
{\sc Theorem 1:} Let $s_1,\ldots,s_n$ be a given RNA sequence.
For any integers $1 \leq i \leq j \leq n$, let
\begin{eqnarray*}
\emZ{i,j} = \sum_{k=0}^n c_k x^k
\end{eqnarray*}
where
\[ c_k(i,j)= \bfZ{k}{i}{j}.
\]
Then  for $i\leq j \leq i+\theta$, $\emZ{i,j}=1$ and for
$j>i+\theta$ we have the recurrence relation
\begin{eqnarray}
\label{eqn:polynomialRecursionNussJac}
\emZ{i,j} &= \emZ{i,j-1} \cdot x^{b_0} +
\sum_{\substack{s_r s_j \in \mathbb{B},\\i\le r<j}}
\left(e^{\frac{-E_0(r,j)}{RT}}\cdot
\emZ{i,r-1}\cdot\emZ{r+1,j-1}\cdot x^{b(r)} \right).
\end{eqnarray}
\medskip
where $b_0 = 1$ if $j$ is base-paired in $\strSt_{[i,j]}$ and $0$ otherwise, and
$b(r)=d_{BP}(\strSt_{[i,j]}, \strSt_{[i,r-1]} \cup \strSt_{[r+1,j-1]} \cup\{(r,j)\})$.

\noindent
{\sc Proof:}
First, some notation is necessary. Recall that if $F$ is an arbitrary
polynomial [resp. analytic] function, then $[x^k] F(x)$
denotes the coefficient of $x^k$ [resp. the $k$th Taylor coefficient in the
Taylor expansion of $F$] -- for instance, in equation~(\ref{eqn:PofX}),
$[x^k]p(x) = p_k$, and in equation~(\ref{eqn:ZofX}), $[x^k]\emZ{} = c_k(i,j)$.

By definition, it is clear that $\emZ{i,j}=1$ if $i\leq j \leq i+\theta$,
where we recall that $\theta = 3$ is the minimum number of unpaired bases in
a hairpin loop.  For $j>i+\theta$, we have
\begin{eqnarray*}
[x^k] \emZ{i,j} &=& c_k(i,j) = \bfZ{k}{i}{j}\\
&=&\bfZ{k-b_0}{i}{j-1} + \sum_{r=i}^{j-1} \sum_{k_0+k_1 = k-b(r)}
e^{\frac{-E_0(r,j)}{RT}} \cdot \bfZ{k_0}{i}{r-1} \cdot \bfZ{k_1}{r+1}{j-1} \\
&=&
[x^{k-b_0}] \emZ{i,j-1} +
\sum_{r=i}^{j-1} \sum_{k_0+k_1 = k-b(r)}
e^{\frac{-E_0(r,j)}{RT}} \cdot \left\{ [x^{k_0}] \emZ{i,r-1} \right\} \cdot
\left\{ [x^{k_1}] \emZ{r-1,j-1} \right\} \\
&=&
[x^{k-b_0}] \emZ{i,j-1} +
\sum_{r=i}^{j-1} \sum_{k_0+k_1 = k-b(r)}
e^{\frac{-E_0(r,j)}{RT}} \cdot [x^{k_0+k_1}] \emZ{i,r-1} \emZ{r-1,j-1} .\\
\end{eqnarray*}
By induction, the proof of the theorem now follows. $\Box$
\medskip

Notice that if one were to compute all terms of the polynomial $\emZ{1,n}$
by explicitly performing polynomial multiplications,
then the computation would require $O(n^5)$ time and $O(n^3)$ space.
Instead of explicitly performing polynomial expansion in {\em variable} $x$,
we instantiate $x$ to a fixed complex number $\alpha \in \mathbb{C}$, and apply
the following recursion for this instantiation:
\begin{eqnarray}
\label{eqn:RNAborNussRecursionInstantiated}
\emZof{i,j}{\alpha} &= &\emZof{i,j-1}{\alpha} \cdot \alpha^{b_0} +
\sum_{\substack{(s_r,s_j) \in \mathbb{B},\\i\le r<j}}
\left(e^{\frac{-E_0(r,j)}{RT}}\cdot
\emZof{i,r-1}{\alpha}\cdot\emZof{r+1,j-1}{\alpha}\cdot \alpha^{b(r)} \right).
\end{eqnarray}
In this fashion, we can compute $\emZof{}{\alpha}=\emZof{1,n}{\alpha}$ in
$O(n^3)$ time and $O(n^2)$ space. For $n$ distinct complex values
$\alpha_0,\ldots,\alpha_{n-1}$, we can compute and save only the
values $\emZof{}{\alpha_0},\ldots, \emZof{}{\alpha_{n-1}}$, each time re-using the
$O(n^2)$ space for the next computation of $\emZof{}{\alpha_k}$. It follows that
the computation resources used to determine the (column) vector
\begin{eqnarray}
\label{eqn:defY}
{\bf Y} = (y_0,\ldots,y_{n-1})^T =
\left(
\begin{array}{l}
y_0\\
y_1\\
\vdots \\
y_{n-1}\\
\end{array}
\right)
\end{eqnarray}
where
$y_0=\emZof{}{\alpha_0},\ldots, y_{n-1}=\emZof{}{\alpha_{n-1}})$ is thus
$O(n^4)$ are quartic time $O(n^4)$ and quadratic space $O(n^2)$.

\subsection*{Polynomial interpolation using the FFT}
Let $\omega = e^{2 \pi i /n}$ be the principal $n$th complex root of unity.
Recall that the
Vandermonde matrix $V_{n}$ is defined to be the
$n \times n$ matrix, whose $i,j$ entry is $\omega^{i \cdot j}$;
i.e.
$$
V_{n} = \left(
\begin{array}{rrrrr}
1&1&1&\cdots&1\\
1&\omega&\omega^2&\cdots&\omega^{n-1}\\
1&\omega^2&\omega^4&\cdots&\omega^{2(n-1)}\\
1&\omega^3&\omega^6&\cdots&\omega^{3(n-1)}\\
\vdots& \vdots& \vdots& \vdots& \vdots\\
1&\omega^{n-1}&\omega^{2(n-1)}&\cdots&\omega^{(n-1)(n-1)}\\
\end{array}
\right)
$$
The Fast Fourier Transform (FFT) is defined to be the $O(n \log n)$
algorithm to compute the Discrete Fourier Transform (DFT), defined
as the matrix product ${\bf Y} = V_{n} {\bf A}$:
$$
\left(
\begin{array}{l}
y_0\\
y_1\\
y_2\\
\vdots \\
y_{n-1}\\
\end{array}
\right)
= V_n \cdot
\left(
\begin{array}{l}
a_0\\
a_1\\
a_2\\
\vdots \\
a_{n-1}\\
\end{array}
\right)
$$
On page 837 of  \cite{cormen}, it is shown that the
$(i,j)$ entry of $V_n^{-1}$ is $\frac{\omega^{-j i}}{n}$
and that
\begin{eqnarray}
\label{eqn:ajdef}
a_j &=&\frac{1}{n} \sum_{k=0}^{n-1} y_k \omega^{-kj}
\end{eqnarray}
for $j=0,\ldots,n-1$.

Since we defined $\bf Y$ in (\ref{eqn:defY}) by ${\bf Y} =
(y_0,\ldots,y_{n-1})^T$, where
$y_0=\emZof{}{\alpha_0},\ldots, y_{n-1}=\emZof{}{\alpha_{n-1}})$ and $\alpha_k = \omega^k
\exp(\frac{k \cdot 2\pi i}{n})$, it follows that the coefficients
$c_k=\bfZ{k}{1,n}$ in the polynomial
$\emZ{} = c_0 + c_1 x + \cdots + c_{n-1} x^{n-1}$ defined in
(\ref{eqn:polynomialYann}) can be computed, at least in principle,
by using the FFT. It turns out, however, that the values of
\bfZ{k}{1,n} are so astronomically large, that the ensuing numerical
instability makes even this approach infeasible for values of $n$
that exceed $56$ (data not shown).
Nevertheless, our approach can be modified as follows.
Define $\bf Y$ by ${\bf Y} = (y_1,\ldots,y_n)^T$, where
$y_1=\frac{\emZof{}{\alpha_1}}{\emZ{}},\ldots, y_{n}=\frac{\emZof{}{\alpha_n-1}}{\emZ{}}$, and
\bfZ{}{} is the partition function defined in (\ref{eqn:mcCaskillPartitionFun}).
Using the FFT to compute the inverse DFT, it follows from
(\ref{eqn:ajdef}) that we can compute the probabilities $p_0,\ldots,p_{n-1}$
that are coefficients of the polynomial
$p(x)=p_0+p_1 x + \cdots + p_{n-1}x^{n-1}$
defined in equation~(\ref{eqn:PofX}). For genomics applications, we are
only interested in the $m$ most significant digits of each $p_k$, as described
in the pseudocode below.
\medskip

\noindent
{\sc Algorithm} for \fftbor\hfill\break
This pseudocode computes the $m$ most significant digits
of probabilities $p_k = \frac{\bfZ{k}{1,n}}{\bfZ{}{}}$. \hfill\break
{\sc Input:} RNA sequence $\seq=s_1,\ldots,s_n$, and initial secondary
structure \strSt of \seq, and integer $m$. \hfill\break
{\sc Output:} Probabilities $p_k=\bfZ{k}{1,n}/\bfZ{}{}$ to $m$ significant digits for $k=0,\ldots,n-1$. \hfill\break
\smallskip
% \mverbatim
% 1.  generate roots of unity $\omega^k$ for $k=0,\ldots,n-1$, where $\omega=\exp(\frac{2 \pi i}{n})$ and $i=\sqrt{-1}$
% 2.  note that the partition function $\bfZ{}{}=y_0=\Z(\omega^0)$
% 3.  for $k=0$ to $n-1$
% 4.    compute $y_k = \Z(\omega^k)$ using recursion (\ref{eqn:RNAborNussRecursionInstantiated})
% 5.    $y_k = 10^m \cdot \frac{y_k}{\bfZ{}{}}$ //normalize $y_k$
% 6.  compute $P = (p_0,\ldots,p_{n-1})^T$ where $p_j =\frac{\sum_{k=0}^{n-1} a_k \omega^{-kj}}{n}$ by using FFT in (\ref{eqn:ajdef})
% 7.  for $k=0$ to $n-1$
% 8.    $p_k = \lfloor 10^m \cdot p_k \rfloor \cdot \frac{1}{10^m}$
% 9.    //truncate to $m$ most significant digits
% |mendverbatim
\medskip

\subsubsection*{Speed-up in our implementation of \fftbor}

In this subsection, we show that we need only evaluate the polynomial
\emZ{}, as defined in
equation~(\ref{eqn:polynomialYann}), for $n/2$ of the complex $n$th roots
of unity. It is first necessary to recall the definition of complex
conjugate.
Recall that the complex conjugate of $z$ is denoted by $\overline{z}$;
i.e. if $z=a+bi$ where $a,b \in \mathbb{R}$ are real numbers and
$i = \sqrt{-1}$,  then $\overline{z} = a-bi$.
\smallskip

\noindent
{\sc Lemma 1:} If \emZ{} is the complex polynomial defined in
equation~(\ref{eqn:polynomialYann}), then for any complex $n$th root of
unity $\alpha$, it is the case that $\emZof{}{\overline{\alpha}} =
\overline{\emZof{}{\alpha}}$. In other words, if $\alpha$ is a complex $n$th root
of unity of the form $a+bi$, where $a,b \in \mathbb{R}$ and $b>0$, and
if $\emZof{}{a+bi} = A + Bi$ where $A,B \in \mathbb{R}$, then it is the case that
\[
\emZof{}{a-bi} = A - Bi.
\]
\smallskip

\noindent
{\sc Proof:}
Letting $i = \sqrt{-1}$, if  $\theta = \frac{2 \pi}{n}$, then
$\omega = e^{i \theta} = \cos(\theta) + i \sin(\theta)$
is the principal $n$th complex root of unity, and
$1=\omega^{0},\ldots,e^{(n-1)\cdot i \theta}=\omega^{n-1}$ together
constitute the complete collection of all
$n$th complex roots of unity -- i.e. the $n$ unique solutions of
of the equation $x^n -1 = 0$ over the field $\mathbb{C}$ of complex numbers.
Clearly, for any $1 \leq r < n$,
$e^{-i r \theta} = 1 \cdot e^{-i r \theta} =
e^{2 \pi i} \cdot e^{-i r \theta} = e^{i(2 \pi - r \theta)} =
e^{i(n \theta - r \theta)} = e^{i \theta (n - r)}$.
Moreover, if $\omega^r = e^{i r \theta} = a + b i$ where
$b>0$, then we have $e^{-i r \theta} = a-bi$. It follows that for any
$n$th root of unity of the form $a+bi$, where $b>0$, the number $a-bi$
is also an $n$th root of unity.

Recall that $\emZ{} = \sum_{k=0}^n c_k x^k$, where
$c_k\in \mathbb{R}$ are real numbers representing the partition function
\bfZ{k}{1,n} over
all secondary structures of a given RNA sequence $s_1,\ldots,s_n$,
whose base pair distance from initial structure
\strSt is $k$. Thus, in order to prove the lemma, it suffices to show
that for all values $k=0,\ldots,n-1$, if $a+bi$ is a complex $n$th
root of unity, where $a,b \in \mathbb{R}$
and $b>0$, and if $(a+bi)^k = C+Di$ where $C,D \in \mathbb{R}$, {\em then}
$(a-bi)^k = C-Di$. Indeed, we have the following.

\begin{eqnarray*}
(a+bi)^m &=& \sum_{k=0}^m {m \choose k} a^{m-k}\cdot  (bi)^k \\
(bi)^k  &=& \left\{
\begin{array}{ll}
b^k &\mbox{if $k \equiv 0 \bmod 4$}\\
i b^k &\mbox{if $k \equiv 1 \bmod 4$}\\
-b^k &\mbox{if $k \equiv 2 \bmod 4$}\\
-i b^k &\mbox{if $k \equiv 3 \bmod 4$}\\
\end{array} \right.
\end{eqnarray*}
\begin{eqnarray*}
(a-bi)^m &=& \sum_{k=0}^m {m \choose k} a^{m-k} \cdot (-bi)^k \\
(-bi)^k &=& \left\{
\begin{array}{ll}
b^k &\mbox{if $k \equiv 0 \bmod 4$}\\
-ib^k &\mbox{if $k \equiv 1 \bmod 4$}\\
-b^k &\mbox{if $k \equiv 2 \bmod 4$}\\
ib^k &\mbox{if $k \equiv 3 \bmod 4$}\\
\end{array} \right.
\end{eqnarray*}
It follows that each term of the form
$a^{m-k} \cdot (bi)^k$, for $k=0,\ldots,m$, is the complex conjugate of
$a^{m-k} \cdot (-bi)^k$, and thus $(a+bi)^m$ is the complex conjugate of
$(a-bi)^m$. Since \emZof{}{a+bi} is a sum of terms of the form $c_k (a+bi)^k$,
it follows that \emZof{}{a-bi} is the complex conjugate of \emZof{}{a+bi}.
This completes the proof of the lemma. \hfill  $\Box$
\medskip

Lemma 1 immediately entails that we need only evaluate \emZ{} on $n/2$
many of the complex $n$th roots of unity -- namely, those of the form
$a+bi$, where $b \geq 0$. The remaining values of \emZ{} are obtained by
taking conplex conjugates of the first $n/2$ values. This, along with a
precomputation of powers of the complex $n$th roots of unity, leads to an
enormous performance speed-up in our implementation of \fftbor.

\subsection{Recursions to compute the polynomial
\texorpdfstring{\emZ{i,j}}{}}
\label{subsec:fftbor:polynomial}

\subsection{Polynomial interpolation to evaluate
\texorpdfstring{\emZ{i,j}}{}}
\label{subsec:fftbor:fft}

\section{Coarse-grained kinetics with \fftbor}
\label{sec:fftbor:kinetics}

\section{Riboswitch detection with \fftbor}
\label{sec:fftbor:rb}

\section{Benchmarking and performance considerations}
\label{sec:fftbor:benchmarking}
