%!TEX root = ../main.tex

\chapter{FFTbor}
\label{ch:fftbor}

\lhead{FFTbor}

\section{Introduction}
\label{sec:fftbor:intro}

In this chapter, we present the \fftbor algorithm and accompanying software.
\fftbor is a novel algorithm developed with the intent of efficiently computing
the Boltzmann probability of those structures whom, for a given input RNA
sequence \seq, differ by $k$ base pairs. By leveraging polynomial interpolation
via the \fft, this algorithm runs in \On{4} time and
\On{2} space, a significant improvement over its predecessor. The accompanying
software which implements this algorithm has been used to predict the location
of expression platforms for putative \rbs in genomic data, and to
evaluate the correlation between kinetic folding speed and landscape ruggedness.

\subsection{Organization}
\label{subsec:fftbor:org}

This chapter is organized in the following fashion. First, we provide
background on
the problem which \fftbor aims to address, as well as a brief overview of
existing approaches. We follow by a formal explanation of the problem, and
proceed to describe how the energy landscape is coarsified into discrete bins.
We then develop the recursions for the parameterized partition function using
the Nussinov-Jacobson energy model, which allows us to exposé the novel aspects
of the algorithm. After developing the recursions, we indicate how they can be
reformulated as a polynomial whose coefficients $c_k=\bfZ{k}{1,n}$. We then
describe how the \fft can be employed to efficiently compute the coefficients
$c_k$, finishing our description of the underlying algorithm. Then we proceed
to present two applications of \fftbor, an application to RNA kinetics and
another to \rb detection in genomic data. Finally, we give reference
to the full recursions using the more accurate Turner energy model, which the
underlying implementation actually uses.

\section{Background}
\label{sec:fftbor:bkgrnd}

\section{Formalization of the problem}
\label{sec:fftbor:formal}

\fftbor aims to compute the coefficients $p_0,\ldots,p_{n-1}$ of the polynomial
\begin{align}
\label{eq:fftbor:pOfX}
p(x) = p_0 + p_1 x + p_2 x^2 + \cdots + p_{n-1} x^{n-1},
\end{align}
where $p_k$ is defined as $p_k = \frac{Z_k}{Z}$. We employ the \fft to compute
the \idft on values $y_0,\ldots,y_{n-1}$, where
$y_k = p(\omega^k)$ and $\omega = e^{2 \pi i/n}$ is the principal $n$th complex
root of unity and $p(x)$ is defined in (\ref{eq:fftbor:pOfX}). By leveraging
complex
roots of unity in conjunction with the \idft the we subvert numeric instability
issues observed with both Lagrange interpolation and Gaussian elimination.

Consider an RNA sequence $\seq = s_1,\ldots,s_n$, where
$s_i \in \{A,C,G,U\}$, i.e. a sequence of nucleotides. We can describe a
secondary structure \str which is compatible with \seq as a collection of
base pair tuples $(i,j)$, where $1 \le i \le i+\theta < j \le n$ and
$\theta \ge 0$---generally taken to be 3, the minimum number of unpaired bases
in a hairpin loop due to steric constraints.

To more simply develop the underlying recursions for \fftbor, we introduce a
number of constraints on the base pairs within \str. Firstly, we require that
each base pair is either a Watson-Crick or G-U wobble, i.e. base pair $(i,j)$
for sequence \seq has corresponding nucleotides $(s_i,s_j)$, which are
restricted to the set
\[
\mathbb{B} = \{(A,U), (U,A), (G,C), (C,G), (G,U), (U,G)\}.
\]
With
this constraint satisfied we say that \str is {\em compatible} with \seq, and
for the remainder of this chapter will only consider those structures which are
compatible with \seq.
Secondly, we insist that given two base pairs $(i,j), (x,y)$ from \str,
$i=x \iff j=y$---bases have at most one partner. Finally, we require that
$i<x<j \iff i<y<j$, no pseudoknots are allowed. While pseudoknots have been
shown to be present in some biologically relevant RNAs, their inclusion greatly
complicates the recursive decomposition of the structure, and thus it is common
to ignore them.

Provided two secondary structures $\str,\strT$, we can define a notion of
distance between them. There are a number of different definitions of distance
used across the literature; we will use {\em \bpd} for \fftbor.
Base pair distance is defined as the symmetric difference between the sets
\str,\strT
\begin{align}
\label{eq:fftbor:dBP}
\dBP{\str,\strT} = |\str \cup \strT| - |\str \cap \strT|.
\end{align}
Given this definition of distance, two structures \str and \strT are said to
be \kNbrs if $\dBP{\str,\strT} = k$. It is important to note that
the notion of \bpd is also applicable to restrictions of secondary structures
on the subsequence $\seq_{i,j}$,
i.e. $\str_{i,j} = \{ (x,y) \,:\, i \leq  x < y \leq j,  (x,y) \in \str \}$.

For a restriction of base pairs for a given structure $\str_{i,j}$,
$\strT_{i,j}$ is said to be a \kNbr of $\str_{i,j}$ if
\[
\dBP{\str_{i,j},\strT_{i,j}} = |\{ (x,y): i \leq x<y\leq j,
(x,y) \in \str - \strT \mbox{ or } (x,y) \in \strT - \str \}| = k.
\]

\section{Derivation of the \fftbor algorithm}
\label{sec:fftbor:math}

Given an RNA sequence $\seq=s_1,\ldots,s_n$ and compatible secondary structure
\strSt, let $\bfZ{k}{}$ denote the sum of the Boltzmann factors
\boltzf{\str} of all $k$-neighbors \str of \strSt; i.e.
\[
\bfZ{k}{} = \bfZ{k}{1,n} =
\sum_{\substack{\str \text{ such that }\\ \dBP{\str,\strSt}=k}}
\boltzF{\str}
\]
where $E(\str)$ denotes the Turner (nearest neighbor)
energy \cite{}
of \str, $R = 0.00198$ kcal/mol denotes the universal
gas constant and $T$ denotes absolute temperature. From this, it follows that
the full partition function is defined as
\begin{align}
\label{eq:fftbor:partFunc}
\bfZ{}{} = \bfZ{}{1,n} = \sum_{k=0}^n \bfZ{k}{1,n}
\end{align}
since the \bpd between \strSt and \str is at most
\begin{align}
\label{eq:fftbor:maxDist}
\dBP{\strSt,\str} \leq |\strSt| + \lfloor \frac{n-\theta}{2} \rfloor \leq n.
\end{align}

We can then define the Boltzmann probability of all \kNbrs \. of \strSt as
\begin{align}
\label{eq:fftbor:probK}
p(k) =\frac{\bfZ{k}{1,n}}{\bfZ{}{1,n}}.
\end{align}

By visualizing the probabilities $p_k$ as a function of $k$, we generate a
coarse grained view of the one-dimensional energy landscape of \seq with
respect to \strSt. When \strSt is taken to be the \mfes for example, one would
anticipate to see a peak at $k=0$, with additional peaks implying additional
metastable structures; local energy minima which could suggest an energetic
trap while folding.

\subsection{Definition of the partition function
\texorpdfstring{\bfZ{k}{1,n}}{}}
\label{subsec:fftbor:recursions}

\subsection{Recursions to compute the polynomial
\texorpdfstring{\emZ{i,j}}{}}
\label{subsec:fftbor:polynomial}

\subsection{Polynomial interpolation to evaluate
\texorpdfstring{\emZ{i,j}}{}}
\label{subsec:fftbor:fft}

\section{Coarse-grained kinetics with \fftbor}
\label{sec:fftbor:kinetics}

\section{Riboswitch detection with \fftbor}
\label{sec:fftbor:rb}

\section{Benchmarking and performance considerations}
\label{sec:fftbor:benchmarking}
