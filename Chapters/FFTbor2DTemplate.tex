%!TEX root = ../main.tex

\newcommand{\p}{{\bf p}}

\chapter{FFTbor2D} % Main chapter title

\label{FFTbor2D} % Change X to a consecutive number; for referencing this chapter elsewhere, use \ref{ChapterX}

\lhead{Chapter X. \emph{FFTbor2D}} % Change X to a consecutive number; this is for the header on each page - perhaps a shortened title

\section{Introduction}

RNA folding pathways play an important role in biological processes.
For instance, in the {\em hok/sok}
(host-killing/suppression of killing) system \citep{Gerdes.arg97},
the transition between two metastable RNA structures determines the
fate of a cell as follows.
The {\em hok} gene of {\em E. coli} and other bacteria
codes a small (52 amino acid) toxin causing irreversible damage to the cell
membrane. It has been shown that {\em hok}-mRNA is
constitutively expressed from a weak promoter, while
the rapidly degraded {\em sok}-RNA is constitutively expressed from
a strong promoter.  The {\em hok}-mRNA is initially
inactive, since a foldback sequesters the
Shine-Dalgarno sequence; however, slow exonucleolytic processing
digests the last $\approx 40$ nt of the $3'$ end of {\em hok}-mRNA,
transforming the molecule into its active form in which
the Shine-Dalgarno sequence is no longer sequestered.
If R1 plasmids of {\em E. coli} are present in
sufficient copy number, then a portion of the 64 nt
{\em sok}-RNA, which is complementary to {\em hok}-mRNA leader
region, binds to the active conformation of {\em hok}-mRNA, thus
causing degradation of the complex by RNase III \citep{Gerdes.arg97}.
If plasmids are not present in sufficient copy number, then the
cell is killed by {\em hok} toxin, thus ensuring fitness of the daughter
cells.

In the case of spliced leader (SL) RNA from certain trypanosomes and nematodes,
a portion of the $5'$ exon is donated to
another mRNA by trans splicing.
Intermediate structures appear to be important in the process of splicing,
as shown by LeCuyer and Crothers \citep{lecuyerCrothers}, who performed
stopped-flow rapid-mixing and temperature-jump measurements
of the kinetics for the structural transition between two low
energy structures of SL RNA from {\em Leptomonas collosoma}.
Conformational switches are thought not only to play a role in such
trans splicing, but as well in transcriptional and
translational regulation, protein synthesis, and mRNA splicing.

For these reasons, substantial experimental and computational work
has been done on folding pathways and the kinetics of RNA folding;
below, we cite only a small sample of the work in this area.
On the experimental side,
%Baird et al.  \citep{Baird.jacs10} used cryo-electron microscopy and single
%particle image reconstruction to determine the structure of the major folding
%intermediate of the specificity domain of a ribonuclease P ribozyme.
%Mitra et al.  \citep{Mitra.r11} conducted experiments using
%time-resolved hydroxyl radical probing of backbone solvent accessible
%surface and catalytic activity measurements, integrated with
%structural-kinetic modeling to compare folding pathways of
%three group I intron ribozymes.
Neupane et al. \citep{Neupane.nar11} applied single-molecule force spectroscopy
of the add adenine riboswitch, to show how folding  relates to
gene regulation;
see also \citep{Woodside.cocb08,Baird.jacs10,Mitra.r11}.
%see \citep{Woodside.cocb08} for a recent comprehensive review of such methods.
%
On the computational side, Morgan and Higgs \citep{morganHiggsBarrier}
appear to have been among the first
to have considered the computational problem of determining optimal and
near-optimal {\em folding pathways} between two metastable secondary
structures $A,B$ of a given RNA sequence.  In \citep{flammPhD}, Flamm
developed an event-driven Monte Carlo simulation, \kinfold,
to determine expected
folding time between two reference structures, including the mean first
passage time to fold into the minimum free energy structure.
%In \citep{Xayaphoummine.nar05}, Xayaphoummine et al. described the
%{\tt Kinefold} web server, for Monte Carlo kinetic folding of an RNA sequence
%into a metastable pseudoknotted structure.
Using \rnasub  \citep{Wuchty.b99}, Flamm et al.
\citep{flamm} designed the exponential
time exact algorithm, {\tt barriers}, that computes the optimal folding
pathway between metastable structures \strA and \strB. Since {\tt barriers}
may not converge, due to its reliance on the enumeration of possibly
exponentially many structures, in \citep{Dotu.nar10} we developed a
local search (tabu) method that rapidly returns near-optimal folding
pathways; in benchmarking results \citep{Dotu.nar10}, local search was shown
to outperform various methods, including direct and indirect path methods
of Morgan and Higgs \citep{morganHiggsBarrier}, breadth-first search with
bounded lookahead, {\tt Findpath} \citep{Flamm.r01}, and the
exponential time, exact method {\tt barriers} \citep{flamm}, in producing
near-optimal pathways within
reasonable time.
There are far too many contributions to kinetics and folding pathways
to adequately survey here; the references
\citep{Xayaphoummine.nar05,Tang.jmb08,Geis.jmb08,Zhao.bj10,Li.bb12}
give some idea of the variety of methods.
%\citep{Xayaphoummine.nar05,Geis.jmb08,Cao.bj09,Zhao.bj10,Zhao.jcp11,Li.bb12}.
%See Abfalter et al. \citep{abfalterFlammStadler} and Flamm et al.
%\citep{flamm} for methods to computationally design bistable switches.

\subsection{Preliminaries}


A secondary structure for a given RNA nucleotide sequence
$\seq = s_1,\dots,s_n$ is a set $S$ of Watson-Crick or wobble
base pairs $(i,j)$, containing neither base triples nor pseudoknots.
%Peter commented out
%A secondary structure for a given RNA nucleotide sequence
%$\seq = s_1,\dots,s_n$ is a set $S$ of base pairs $(i,j)$, such that
%{\em (i)} if $(i,j)\in S$ then
%$s_i,s_j$ form either a Watson-Crick (AU,UA,CG,GC) or
%wobble (GU,UG) base pair,
%{\em (ii)} if $(i,j)\in S$ then $j-i>\theta=3$ (a steric constraint
%requiring that there be at least $\theta=3$ unpaired bases between
%any two paired bases),
%{\em (iii)} if $(i,j)\in S$ then for all $j' \ne j$ and $i' \ne i$,
%$(i',j) \not\in S$ and $(i,j') \not\in S$ (nonexistence of base triples),
%{\em (iv)} if $(i,j)\in S$ and $(k,\ell)\in S$, then
%it is not the case that $i<k<j<\ell$ (nonexistence of pseudoknots).
The number of base pairs in $S$ is denoted by $|S|$.
The secondary structure $S$ is {\em compatible}
with \seq if for every base pair $(i,j)$ in $S$, the pair
$(s_i,s_j)$ is contained in the set
%$\mathbb{B} = \{(A,U), (U,A), (G,C), (C,G), (G,U), (U,G)\}$
$\mathbb{B}$
of six canonical (Watson-Crick and wobble) base pairs.
Throughout this paper, by {\em structure}, we always mean a
secondary structure which is compatible with an arbitrary, but fixed
RNA sequence \seq.

If $A,B$ are secondary structures of \seq, then the {\em base pair distance},
\dBP{\strA}{\strB}, is defined to be the $|A-B|+|B-A|$, i.e. the number of
base pairs belonging to one structure but not the other.
Structures $S,T$ are said to be $k$-{\em neighbors} if $d_{BP}(S,T) = k$.
In \citep{Freyhult.b07}, we described recursions for an $O(n^5)$ time and
$O(n^3)$ algorithm, {\tt RNAbor}, that computes the Boltzmann probability
$p_k$ of structures to have base pair distance $k$ from a given reference
structure \strA. In \citep{fftbor}, we described an $O(n^4)$ time and
$O(n^2)$ space algorithm, by using polynomial interpolation with the
fast Fourier transform (FFT).
In \citep{hofacker:RNAbor2D}, Lorenz et al. generalized
the recursions  of {\tt RNAbor} \citep{Freyhult.b07}
to yield recursions for an $O(n^7)$ time and $O(n^4)$ space
algorithm, \rnatwofold,
that computes the Boltzmann probability $p(x,y)$ that a structure has
base pair distance $x$ from reference structure \strA, and distance
$y$ from another reference structure \strB.
The goal of this paper is to describe a new algorithm, \ffttwo,
using polynomial interpolation with the FFT, to
reduce the worst case complexity of \rnatwofold to
$O(n^5)$ time and $O(n^2)$ space. As well, we provide an
illustrative application by computing the mean first passage time
between metastable structures $A,B$ of spliced leader RNA from
{\em L. collosoma} on the 2D energy landscape computed by \ffttwo.

The general idea of using
interpolation to compute partition function values was first suggested
by Waldisp\"uhl and Ponty in the context of the
{\tt RNAmutants} program \citep{waldispuhlPontyRecomb}. Subsequently,
we used the Fast Fourier Transform (FFT) in our algorithm
\fftbor \citep{fftbor}  to interpolate the
probabilities $p_k$ that structures from the Boltzmann
ensemble have base pair distance $k$ from a target structure \strSt.
This paper extends the result from \citep{fftbor} to two dimensions; i.e.
in the algorithm \ffttwo,
we interpolate the probabilities $p(x,y)$ that
structures from the Boltzmann ensemble have base pair distance
$x$ [resp. $y$] from target structure \strA [resp. \strB].

\subsection{Plan of paper}

The plan for the rest of this paper is as follows. In
Section~\ref{section:approach}, we develop the quintic
$O(n^5)$ time and quadratic $O(n^2)$ space algorithm \ffttwo,
which uses dynamic programming to evaluate a complex polynomial
$\Z(x)$ at quadratically many complex roots of unity, and then use the fast
Fourier transform (FFT) to compute the coefficients of $\Z(x)$ by
polynomial interpolation. The coefficients of $\Z(x)$ yield the 2D
energy landscape for a given RNA sequence. Moreover, by exploiting
parity and complex conjugates, we obtain an additional reduction of time
by a factor of $4$. Although this section is rather technical, the
main result entails a significant improvement in the algorithm
run time.
In Section~\ref{section:benchmarking}, we present bencharking results,
comparing the run times of \ffttwo and \rnatwofold
(developed by Lorenz et al.  \citep{hofacker:RNAbor2D}).
In Section~\ref{section:kinetics}, we apply \ffttwo to determine
the mean first passage time (MFPT) along the 2D energy grid in folding
between two metastable structures of {\em L. collosoma} spliced leader
RNA.
Finally, in Secton~\ref{section:discussion}, we we describe
differences in the algorithms \rnatwofold and
and \ffttwo, and mention relative strengths of each software
depending on the envisioned application.

\section{Polynomial interpolation using the FFT}
\label{section:approach}

For expository clarity, we describe \ffttwo and all recursions
in terms of the Nussinov energy model \citep{nussinovJacobson}, where
the energy $E_0(i,j)$ of a base pair $(i,j)$ is defined to be $-1$, and the
energy $E(S)$ of a secondary structure $S$ is $-1$ times the number $|S|$
of base pairs in structure $S$.  Nevertheless, the implementation of
\ffttwo involves the full Turner energy model \citep{xia:RNA}, where
free energy $E(S)$ depends on negative, stabilizing energy contributions
from base stacking, and positive, destabilizing energy contributions due to
loss of entropy in loops.

Given reference secondary structures $A,B$ of a
given RNA sequence $\seq=s_1,\ldots,s_n$, our goal is to compute
\begin{align}
\label{eq:defZk}
\z{x,y}{1}{n} &= \sum_{\substack{S \text{ such that }\\
d_{BP}(S,A)=x, d_{BP}(S,B)=y}} e^{\frac{-E(S)}{RT}}
\end{align}
for all $0\leq x,y < n$, where $R$ is the universal gas constant, $T$
absolute temperature, $E(S)$ denotes the free energy of $S$, and $S$ ranges
over all secondary structures that are compatible with \seq. As mentioned,
we emphasize that for expository reasons alone, the Nussinov energy model is
used in the recursions in this paper, although full recursions and
the implementation of \ffttwo involve the Turner energy model.

For any secondary structure $S$ of \seq, and any values
$1\leq i\leq j \leq n$, the restriction $S_{[i,j]}$ is defined to be the
collection of base pairs of $S$, lying within interval $[i,j]$; i.e.
$S_{[i,j]} = \{ (k,\ell) : i \leq k < \ell \leq j\}$.
In \citep{hofacker:RNAbor2D}, Lorenz et al. generalized
the dynamic programming recursions of our earlier work \citep{Freyhult.b07},
to yield recursions
for the partition function $\z{x,y}{i}{j}$ in equation
(\ref{eq:defZk}).  In the context of the Nussinov model,
$\z{x,y}{i}{j}$ is equal to
\begin{eqnarray}
\label{eq:RNAborNussRecursion}
&\z{(x-\alpha_0),(y-\beta_0)}{i}{j-1} +  \\
&\sum_{\substack{s_k s_j \in \mathbb{B},\\i\le k<j}}
\left(e^{\frac{-E_0(k,j)}{RT}}
\sum_{u+u'=x-\alpha(k)} \sum_{v+v'=y-\beta(k)}
\z{u,v}{i}{k-1}  \z{u',v'}{k+1}{j-1} \right) \nonumber
\end{eqnarray}
where $\alpha_0 = 1$ if $j$ is base paired in $A_{[i,j]}$ and $0$ otherwise,
$\beta_0 = 1$ if $j$ is base paired in $B_{[i,j]}$ and $0$ otherwise,
$E_0(k,j)=-1$ if $k,j$ can base-pair, and otherwise $E_0(k,j)=0$, and
$\alpha(k)=d_{BP}(A_{[i,j]}, A_{[i,k-1]} \cup A_{[k+1,j-1]} \cup\{(k,j)\})$,
and
$\beta(k)=d_{BP}(B_{[i,j]}, B_{[i,k-1]} \cup B_{[k+1,j-1]} \cup\{(k,j)\})$.

\subsection{Recursions to compute the polynomial $\Z_{i,j}(x)$}
\label{section:recursionsForPolynomialZij}

Given RNA sequence $\seq = s_1,\ldots,s_n$
and two arbitrary, but fixed reference
structures $A,B$, we define the {\em polynomial}
\begin{eqnarray}
\label{eq:polynomialYann}
\Z(x) = \sum_{r=0}^{n-1} \sum_{s=0}^{n-1}  z_{rn+s} x^{r\cdot n + s}
\end{eqnarray}
where (constant) coefficients
\[ z_{rn+s}= \z{r,s}{1}{n} =
\sum_{\substack{S \text{ such that }\\
d_{BP}(S,A)=r, d_{BP}(S,B)=s}}
e^{\frac{-E(S)}{RT}}
\]
where $E(S)$ denotes the free energy of $S$.
If we evaluate the polynomial $\Z(x)$ at $n^2$ distinct pairs of values
$a_0,\ldots,a_{n^2-1}$ in
\begin{eqnarray}
\label{eq:ZofX}
\Z(a_{0}) = z_{0},\ldots, \Z(a_{n^2-1}) = z_{n^2-1},
\end{eqnarray}
%then Lagrange polynomial interpolation  \citep{LagrangeInterpolation}
then Lagrange polynomial interpolation
guarantees that we can determine the coefficients $c_{rn+s}$ of $\Z(x)$,
for $0\leq r,s < n$. Due to technical difficulties concerning numerical
robustness, we will perform polynomial interpolation by using Vandermonde
matrices and the fast Fourier transform (FFT).

The following theorem shows that a
recursion, analogous to equation (\ref{eq:RNAborNussRecursion}),
can be used to compute
the {\em polynomial} $\Z_{i,j}(x)$ defined by
\begin{eqnarray}
\label{eq:defZijpolynomial}
\Z_{i,j}(x) &=& \sum_{r=0}^{n-1} \sum_{s=0}^{n-1}
z_{rn+s}(i,j) x^{rn+s}
=\sum_{k=0}^{n^2-1} z_{k}(i,j) x^{k}
\end{eqnarray}
%\begin{eqnarray}
%\label{eq:defZijpolynomial}
%\Z_{i,j}(x) &=& \sum_{r=0}^{n-1} \sum_{s=0}^{n-1}
%z_{rn+s}(i,j) x^{rn+s}  \nonumber \\
%\Z_{i,j}(x) &=& \sum_{k=0}^{n^2-1} z_{k}(i,j) x^{k}
%\end{eqnarray}
where
\[ z_{rn+s}(i,j)= \z{r,s}{i}{j} =
\sum_{\substack{S \text{ such that }\\
d_{BP}(S,A_{[i,j]})=r, d_{BP}(S,B_{[i,j]})=s}}
e^{\frac{-E(S)}{RT}}.
\]
Here, in the summation, $S$ runs over structures on $s_i,\ldots,s_j$, which
are $r$-neighbors of the restriction $A_{[i,j]}$ of reference structure
\strA to interval $[i,j]$, and simultaneously
$s$-neighbors of the restriction $B_{[i,j]}$ of reference structure
\strB to interval $[i,j]$.
\medskip

\noindent
{\sc Theorem 1:} Let $s_1,\ldots,s_n$ be a given RNA sequence.
For any integers $1 \leq i < j \leq n$, let
\begin{eqnarray*}
\Z_{i,j}(x) = \sum_{r=0}^{n-1} \sum_{s=0}^{n-1} z_{rn+s}  x^{rn+s}
\end{eqnarray*}
where
\[
z_{rn+s}(i,j)= \z{r,s}{i}{j}.
\]
%Peter commented out
%Then  for $i\leq j \leq i+\theta$, $\Z_{i,j}(x)=1$ and for
%$j>i+\theta$ we have the recurrence relation
%given by $\Z_{i,j}(x)$ equals
%\begin{eqnarray}
%\label{eq:polynomialRecursionNussJac}
%\Z_{i,j}(x) &= \Z_{i,j-1}(x) \cdot x^{\alpha_0n+\beta_0} + \\
%&\sum_{\substack{s_k s_j \in \mathbb{B},\\i\le k<j}}
%\left(e^{\frac{-E_0(k,j)}{RT}}\cdot
%\Z_{i,k-1}(x)\cdot\Z_{k+1,j-1}(x)\cdot x^{\alpha(k)n+\beta(k)} \right)
%\nonumber
%\end{eqnarray}
%end of Peter
Inductively we define $\Z_{i,j}(x)$ to equal
\begin{eqnarray}
\label{eq:polynomialRecursionNussJac}
&\Z_{i,j-1}(x) \cdot x^{\alpha_0n+\beta_0} + \\
&\sum_{\substack{s_k s_j \in \mathbb{B},\\i\le k<j}}
\left(e^{\frac{-E_0(k,j)}{RT}}\cdot
\Z_{i,k-1}(x)\cdot\Z_{k+1,j-1}(x)\cdot x^{\alpha(k)n+\beta(k)} \right)
\nonumber
\end{eqnarray}
where $\alpha_0 = 1$ if $j$ is base-paired in $A_{[i,j]}$ and $0$ otherwise,
$\beta_0 = 1$ if $j$ is base-paired in $B_{[i,j]}$ and $0$ otherwise, and
$\alpha(k)=d_{BP}(A_{[i,j]}, A_{[i,k-1]} \cup A_{[k+1,j-1]} \cup\{(k,j)\})$,
$\beta(k)=d_{BP}(B_{[i,j]}, B_{[i,k-1]} \cup B_{[k+1,j-1]} \cup\{(k,j)\})$.
The proof is given in supplementary information.
\medskip

Note that if one were to compute all terms of the polynomial $\Z_{1,n}(x)$
by explicitly performing polynomial multiplications,
then the computation would require $O(n^7)$ time and $O(n^4)$ space, the
same time complexity of \citep{hofacker:RNAbor2D}.
Instead of explicitly performing polynomial expansion in {\em variable} $x$,
we instantiate $x$ to a
complex number $\rho \in \mathbb{C}$, and apply
the following recursion, by setting $\Z_{i,j}(\rho)$ equal to
%\begin{eqnarray}
%\label{eq:RNAborNussRecursionInstantiated}
%\Z_{i,j}(\rho) &= &\Z_{i,j-1}(\rho) \cdot \rho^{\alpha_0n+\beta_0} +
%\sum_{\substack{(s_k,s_j) \in \mathbb{B},\\i\le k<j}}
%\left(e^{\frac{-E_0(k,j)}{RT}}\cdot
%\Z_{i,k-1}(\rho)\cdot\Z_{k+1,j-1}(\rho)
%\cdot \rho^{\alpha(k)n+\beta(k)} \right).
%\end{eqnarray}
\begin{eqnarray}
\label{eq:RNAborNussRecursionInstantiated}
&\Z_{i,j-1}(\rho) \cdot \rho^{\alpha_0n+\beta_0} + \\
&\sum_{\substack{(s_k,s_j) \in \mathbb{B},\\i\le k<j}}
\left(e^{\frac{-E_0(k,j)}{RT}}\cdot
\Z_{i,k-1}(\rho)\cdot\Z_{k+1,j-1}(\rho)
\cdot \rho^{\alpha(k)n+\beta(k)} \right) \nonumber
\end{eqnarray}
In this fashion, we can compute $\Z(\rho)=\Z_{1,n}(\rho)$ in
$O(n^3)$ time and $O(n^2)$ space. For $n^2$ distinct complex numbers
$\rho_i$ where $0 \leq i \leq n^2-1$, we can compute and save only the
values $\Z(\rho_0),\ldots, \Z(\rho_{n^2-1})$, each time re-using the
$O(n^2)$ space for the next computation of $\Z(\rho_i)$.
It follows that
the computation resources used to determine the (column) vector
\begin{eqnarray}
\label{eq:defY}
{\bf Y} = (y_0,\ldots,y_{n^2-1})^T =
\left(
\begin{array}{l}
y_0\\
y_1\\
\vdots \\
y_{n^2-1}\\
\end{array}
\right)
\end{eqnarray}
where
$y_0=\Z(\alpha_0),\ldots, y_{n^2-1}=\Z(\alpha_{n^2-1})$ are thus
quintic time $O(n^5)$ and quadratic space $O(n^2)$.

\subsection{Polynomial interpolation}
Our plan is to determine the coefficients of the polynomial
$\Z(x)$ in equation (\ref{eq:polynomialYann}) by
polynomial interpolation.  For reasons of numerical stability,
we instead determine the coefficients of the polynomial $\p(x)$,
defined by
\begin{eqnarray}
\label{eq:polynomialYannP}
\p(x) =
\sum_{r=0}^{n-1} \sum_{s=0}^{n-1}  \p_{rn+s}
 x^{r\cdot n + s} =
%\sum_{r=0}^{n-1} \sum_{s=0}^{n-1}  \frac{z_{rn+s}}{{\bf Z}}
\sum_{r=0}^{n-1} \sum_{s=0}^{n-1}  \frac{z_{rn+s}}{Z}
 x^{r\cdot n + s},
\end{eqnarray}
where the fast Fourier transform (FFT) is used to implement the
interpolation of
the coefficients using the inverse discrete Fourier transform (DFT), as
described in Section~\ref{section:FFT}.  The following pseudocode describes how
to compute the $m$ most significant digits
for probabilities
$p_{rn+s} = \frac{\z{r,s}{1}{n}}{{\bf Z}}$. It is well-known that
the FFT requires $O(N \log N)$ time to solve the inverse discrete
Fourier transform for a polynomial of degree $N$. In our case,
$N=n^2$, and so line 6 involving the FFT requires time $O(n^2 \log n)$.

The pseudocode for the algorithm to compute $\p(x)$ is given in
%Figure \ref{fig:fftbor}.
Figure 1.
In the next section, we explain a highly non-trivial improvement of
this algorithm to reduce time by a factor of $4$.

\begin{figure}[!ht]
\begin{small}
{\sc Algorithm} for \ffttwo\hfill\break
{\sc Input:} RNA sequence $\seq=s_1,\ldots,s_n$, and distinct secondary
structures $A,B$ of \seq, and integer $m$. \hfill\break
{\sc Output:} Probabilities $p_{rn+s} = p(r,s)=\z{r,s}{1}{n}/{\bf Z}$
to $m$ significant digits for $x,y=0,\ldots,n-1$.
Let $i$ denote $\sqrt{-1}$, $\alpha = \exp(\frac{2\pi i}{n^2})$ and
$\alpha^k = \exp(\frac{2\pi i k}{n^2})$.
\end{small}
\hfill\break
\smallskip
\begin{small}
\mverbatim
 1.  for $k=0,\ldots,n^2-1$
 2.    compute the $k$th roots of unity $\alpha^k$
 3.  for $k=0,\ldots,n^2-1$
 4.    compute $y_k = \Z(\alpha^k)$
 5.    $y_k = 10^m \cdot \frac{y_k}{{\bf Z}}$ //normalize $y_k$
 6.  compute $p_0,\ldots,p_{n^2-1}$ by FFT
 7.  for $k=0$ to $n^2-1$
 8.    $p_k = \lfloor 10^m \cdot p_k \rfloor \cdot \frac{1}{10^m}$
 9.  //truncate to $m$ most significant digits
|mendverbatim
\end{small}
\caption{\small
Pseudocode to compute the $m$ most significant digits
for probabilities
$p_{rn+s} = \frac{\z{r,s}{1}{n}}{{\bf Z}}$. In our implementation,
due to numerical stability issues in the FFT engine, precision parameter
$m$ has an upper bound of $8$ -- only the $m=8$ most significant digits
are computed with \ffttwo.
(Note that the software actually usesbase $2$ precision parameter, with maximum of $27$, where $2^{27} \approx
10^8$.)
It is well-known that
the FFT requires $O(N \log N)$ time to solve the inverse discrete
Fourier transform for a polynomial of degree $N$. In our case,
$N=n^2$, and so the FFT requires time $O(n^2 \log n)$.
}
\label{fig:fftbor}
\end{figure}


\subsection{Speed-up by factor of $4$}
Recall that if $a+bi$ is a complex number, where $a,b$ are real values and $i$
denotes $\sqrt{-1}$, then the complex conjugate of $a+bi$, denoted by
$\overline{a+bi}$ is defined to be $a-bi$.  Recall that a complex $n$th
root of unity is a number whose $n$th power equals one. Moreover,
$e^{2 \pi i/n}$ is the {\em principal} complex $n$th root of unity; i.e.
$\{ e^{2 \pi i k/n} : k=0,\ldots,n-1 \}$ is a set of pairwise distinct
$n$th roots of unity. For notational reasons below, we will write
`$n$-root of unity' instead of `$n$th root of unity'.
We have the following.
\medskip

\noindent
{\sc Lemma 1:} Let $A,B$ denote two distinct, arbitrary but fixed,
 secondary structures of RNA sequence \seq, let $S$ range over
all secondary structures of \seq, and let $d_0$ denote \dBP{\strA}{\strB}.
If $x=d_{BP}(A,S)$ and $y=d_{BP}(S,B)$, then
$y \in \{ d_0-x+2k: k=0,\ldots,x\}$.
\medskip

\noindent
It follows that if $x=d_{BP}(A,S)$ and $y=d_{BP}(S,B)$,
then the only possible values for
$(x,y)$ are $(0,d_0), (1,d_0-1), (1,d_0+1),
(2,d_0-2), (2,d_0), (2, d_0+2),
(3,d_0-3), (3,d_0-1), (3, d_0+1), (3,d_0+3), \cdots$.
As a corollary, we have the {\em parity condition}, that
\begin{eqnarray}
\label{eq:parity}
d_{BP}(A,S)+d_{BP}(S,B) \equiv \dBP{\strA}{\strB} \bmod 2
\end{eqnarray}
first noticed in \citep{hofacker:RNAbor2D}, as well as
the triangle inequality
$d_{BP}(A,S)+d_{BP}(S,B) \geq \dBP{\strA}{\strB}$
for base pair distance, probably folklore.
%but explicitly mentioned in \citep{Moulton.jcb00}.
Lorenz et al. \citep{hofacker:RNAbor2D} exploited the parity condition
and the triangle inequality by using sparse matrix methods to improve on
the efficiency of the naive  implementation of the
$O(n^7)$ time and $O(n^4)$ space algorithm to compute
the partition function, $\z{r,s}{1}{n}$,
and minimum free energy structure, $MFE^{r,s}_{1,n}$, over all
structures having base pair distance $r$ to \strA and $s$ to \strB.
The following lemma is not difficult to establish.
\medskip

\noindent
{\sc Lemma 2:} If $\Z(x)$ is the complex polynomial defined in
equation~(\ref{eq:polynomialYann}), then for any complex $n$th root of
unity $\alpha$, it is the case that $\Z(\overline{\alpha}) =
\overline{\Z(\alpha)}$.
\medskip

\noindent
{\sc Lemma 3:}
Let $\Z(x)$ be defined by equation (\ref{eq:polynomialYann}), and
let $\alpha \in \mathbb{C}$ be any complex number.
If the base pair distance between reference
structures $A,B$ is even, then $Z(-\alpha)=Z(\alpha)$, while if
the distance is odd, then $Z(-\alpha)=-Z(\alpha)$.
\medskip

\noindent
{\sc Lemma 4:} Suppose that $M$ is evenly divisible by $4$,
$\nu = \exp(\frac{2 \pi i}{M})$ is the principal $M$-root of unity, and
$\frac{M}{4} < k \leq \frac{M}{2}$. Then
\[
\nu^k = -(\nu^{-(M/2-k)}) = - \overline{\nu^{M/2-k}}.
\]
\medskip

Lemma 1 is proved by simple induction; Lemma 2
is proved by a computation involving binomial coefficients;
Lemma 3 is immediate
by the parity observation above, resulting from Lemma 1;
Lemma 4 is elementary, relying on Euler's
formula and trigonometric addition formulas. Details proofs of
Lemmas 2,3,4 can be found in supplementary information.

Lemma 1 entails that either all even coefficients, or all odd coefficients
of $\Z(x)$ are zero, and so by a variable change described in detail below,
we require only half the number of evaluations of $\Z(x)$, in order to perform
polynomial interpolation.
Lemma 2 entails that we require only half again the number of evaluations of
$\Z(x)$, since the remainder can be inferred by taking the complex conjugate.
Lemma 1 and Lemma 2, along with a
precomputation of powers of the complex roots of unity, lead to a
large performance speed-up in our implementation of \ffttwo -- indeed,
by a factor of $4$ or more. Though the intuitive idea of how to obtain
this speedup by a factor of four may be apparent, the technical details
leading to the pseudocode of \ffttwo, presented in
%Figure \ref{pseudocode:interpolatingP}, are rather tricky. These details
Figure 2, are rather tricky. These details
are presented in the next two subsections, which can be skipped by the
reader wishing to move on to the algorithm itself.
\medskip

\subsection{Time reduction due to Lemma 1}
Let $n$ denote the length of RNA sequence \seq, and let $N$ denote the
least {\em even} integer greater than or equal to $n$. Since $N$ is even,
we have $(r+s) \equiv (r\cdot(N+1)+s) \bmod 2$. For distinct
fixed structures $A,B$, let
$\pi_1(k) = \lfloor \frac{k}{N+1} \rfloor$, and
$\pi_2(k) = k \bmod (N+1)$, and define the polynomial
\begin{eqnarray*}
\label{eq:polynomialPeter}
\Z(x) &=& \sum_{r=0}^{N} \sum_{s=0}^{N} z_{rN+s} x^{r\cdot N + s}\\
&=&
\sum_{k=0}^{(N+1)^2-1} z_{\pi_1(k)\cdot (N+1) + \pi_2(k)}
x^{\pi_1(k)\cdot (N+1) + \pi_2(k)}\\
&=&
\sum_{k=0}^{(N+1)^2-1} z_k x^k
\end{eqnarray*}
where for the last equality, we have used the fact that
$k = \pi_1(k)\cdot (N+1) + \pi_2(k)$, well-known from
row major order of a 2-dimensional array.

Consider the coefficients of the polynomial
\begin{eqnarray}
\label{eq:yannPolynomialN}
\Z(x) = \sum_{r=0}^{N} \sum_{s=0}^{N} z_{rN+s} x^{rN+s}
= \sum_{k=0}^{(N+1)^2-1} z_k x^k.
\end{eqnarray}
Since $N$ is even, the parity of $r+s$ equals the parity of
$r(N+1)+s$, hence  it follows from the parity condition that either
{\em (i)} all coefficients $z_1,z_3,z_5,\ldots$ of odd parity are zero,
or {\em (ii)} all coefficients $z_0,z_2,z_4,\ldots$ of even parity are zero.
To simplify notation, in the remainder of this subsection, let $M$ be
the least integer greater than or equal to $(N+1)^2$ that is evenly divisible
by $4$, and let $M_0=M/2$. We will assume that $\Z(x) = \sum_{k=0}^{M-1}
z_k x^k$, whereupon coefficients $z_k=0$ for $k>(N+1)^2$.
\smallskip

\noindent
{\sc Case 1}: All coefficients $z_k$ of odd parity in
equation (\ref{eq:yannPolynomialN}) are zero.
\smallskip

\noindent
In this case, we have
$\Z(x) = \sum_{k=0}^{\frac{M}{2}-1} z_{2k} x^{2k}$. But then
$\Z(x) = Y(u) = \displaystyle\sum_{k=0}^{M_0-1} b_k u^k$,
where we have made a variable change $u=x^2$, and coefficient changes
$b_k = a_{2k}$.
By evaluating $M_0=\frac{M}{2}$ many complex
$M_0$-roots of unity, we can
use polynomial interpolation to determine all coefficients $b_k$ of
the polynomial
\[
Y(u) = \displaystyle\sum_{k=0}^{M_0-1} b_k u^k =
\displaystyle\sum_{k=0}^{M_0-1} z_{2k} x^{2k}.
\]


Since $Y(x^2)=Z(x)$, we have
$Y(\exp(\frac{2\pi k i}{M/2})) =
Y(\exp(\frac{4\pi k i}{M})) =
Z(\exp(\frac{2\pi k i}{M}))$, hence we
use the previous recursions (\ref{eq:polynomialRecursionNussJac})
to evaluate $Z(\exp(\frac{2\pi ki}{M})$.  Instead of
performing $M$ evaluations of $Z(x)$ at $M$-roots of unity,
this requires only $M_0=M/2$ evaluations of $Y(u)$ at
$M_0$-roots of unity; i.e. only half the number of
evaluations of $Z(x)$ are necessary to
obtain the coefficients of $Y(x)$. But then, we immediately obtain the
full polynomial $\Z(x)$, since its coefficients of odd  parity are zero.
\medskip

\noindent
{\sc Case 2}: All coefficients $z_k$ of even parity in
equation (\ref{eq:yannPolynomialN}) are zero.
\smallskip

\noindent
In this case, $z_0,z_2,z_4,\cdots$ are zero, so
$\Z(x) = \sum_{k=0}^{M/2-1} z_{2k+1} x^{2k+1}$. But then
$\Z(x) = x \cdot Y(u)$, where
$Y(u)= \sum_{k=0}^{M_0-1} b_k u^k$, where we have made
a variable change $u=x^2$, and coefficient changes
$b_k = z_{2k+1}$. Similarly to Case 1,
we can interpolate the $M_0$ coefficients of the polynomial
$Y(u) = \displaystyle\sum_{k=0}^{M_0-1} b_k u^k$
by evaluating $M_0$ many complex $M_0$-roots
of unity.  Since $\Z(x)=x \cdot Y(x^2)$, $Y(x^2) = x^{-1} \cdot \Z(x)$,
so
$Y(\exp(\frac{2\pi ki}{M/2})) = Y(\exp(\frac{4\pi ki}{M})) =
\exp(\frac{-2\pi ki}{M}) \cdot
\Z(\exp(\frac{2\pi ki}{M}))$, employing
the previous recursions (\ref{eq:polynomialRecursionNussJac})
to evaluate $\Z(\exp(\frac{2\pi ki}{M})$.  Note, that unlike the
Case 1, since $Z(x)=x \cdot Y(x^2)$, we have
$Y(x^2)=\frac{\Z(x)}{x}$, which explains the presence of additional factor
$\exp(\frac{-2\pi ki}{M}))$ in Case 2.  Thus, instead of
performing $M$ evaluations of $\Z(x)$ at $M$-roots of unity,
we perform only $M_0=\frac{M}{2}$ evaluations of $Y(u)$ at
$M_0$-roots of unity; i.e. only half the number of
evaluations of $\Z(x)$ are necessary to
obtain the coefficients of $Y(x)$. But then, we immediately obtain the
full polynomial $\Z(x)$, since $\Z(x) = x \cdot Y(x^2)$, and the
coefficients of $\Z(x)$ of even parity are zero.

In the following, we will need the observation, that if the parity of
base pair distance \dBP{\strA}{\strB} between $A,B$ is
even, then
\begin{eqnarray}
\label{eq:evenY}
Y(x^2) = \Z(x)
\end{eqnarray}
while if the parity is odd,
then
\begin{eqnarray}
\label{eq:oddY}
Y(x^2) = \frac{1}{x} \cdot \Z(x).
\end{eqnarray}

\subsection{Time reduction due to Lemma 2}
As before, let $M$ be the the least number evenly divisible by $4$, which is
greater than or equal to $(N+1)^2$, let $\nu = \exp(\frac{2 \pi i}{M})$
and $\omega=\nu^2 = \exp(\frac{2 \pi i}{M})^2 =
\exp(\frac{2 \pi i}{M/2})$. Clearly, $\nu$ is a principal complex
$M$-root of unity, while $\omega$ is a principal complex $\frac{M}{2}$-root
of unity. Evaluate $Z(\alpha)$ for each $\frac{M}{2}$-root of unity
that belongs to the first quadrant, and apply Lemma 2 to infer the values
of $Z(\alpha)$ for each $\frac{M}{2}$-root of unity that belongs to the
fourth quadrant. More precisely,
we compute $Z(\nu^k)$, for $k=0,\ldots,\frac{M}{4}$, and by Lemmas 2,3,4 infer
that for $k=\frac{M}{4}+1,\ldots,\frac{M}{2}-1$, we have
$Z(\nu^{k})= -1^{d_0} \cdot \overline{Z(\nu^{\frac{M}{2}-k})}$, where
$d_0 = \dBP{\strA}{\strB}$. This is justified in the following.

By induction on $k=\frac{M}{4}+1,\ldots,\frac{M}{2}-1$, we have
\begin{eqnarray*}
Y(\omega^k)&=& Y(\nu^{2k})  \nonumber \\
&=& \left\{ \begin{array}{ll}
Z(\nu^{k}) &\mbox{\tiny if
$\dBP{\strA}{\strB} = 0 \bmod 2$}\\
\frac{1}{\nu^{k}} \cdot Z(\nu^{k})
&\mbox{\tiny if
$\dBP{\strA}{\strB} = 1 \bmod 2$}\\
\end{array} \right. \nonumber\\
&=& \left\{ \begin{array}{ll}
Z(- \overline{\nu^{(\frac{M}{2}-k)}}) &\mbox{\tiny if
$\dBP{\strA}{\strB} = 0 \bmod 2$}\\
\nu^{-k} \cdot
Z(-\overline{\nu^{(\frac{M}{2}-k)}})
&\mbox{\tiny if
$\dBP{\strA}{\strB} = 1 \bmod 2$}\\
\end{array} \right. \nonumber\\
&=& \left\{ \begin{array}{ll}
Z(\overline{\nu^{(\frac{M}{2}-k)}}) &\mbox{\tiny if
$\dBP{\strA}{\strB} = 0 \bmod 2$}\\
\nu^{-k} \cdot
-Z(\overline{\nu^{(\frac{M}{2}-k)}})
&\mbox{\tiny if
$\dBP{\strA}{\strB} = 1 \bmod 2$}\\
\end{array} \right.  \nonumber\\
&=& \left\{ \begin{array}{ll}
\overline{Z(\nu^{(\frac{M}{2}-k)})} &\mbox{\tiny if
$\dBP{\strA}{\strB} = 0 \bmod 2$}\\
-\nu^{-k} \cdot
\overline{Z(\nu^{(\frac{M}{2}-k)})}
&\mbox{\tiny if
$\dBP{\strA}{\strB} = 1 \bmod 2$}\\
\end{array} \right.  \nonumber\\
\end{eqnarray*}
Line 1 follows by definition, since $\omega=\nu^2$;
line 2 follows by equations (\ref{eq:evenY}) and
(\ref{eq:oddY});
line 3 follows by Lemma 4;
line 4 follows by Lemma 3.
Thus if \dBP{\strA}{\strB} is even, then
\begin{eqnarray}
\label{eq:trickyRootsOfUnity1}
y_k = Y(\omega^k) = \left\{
\begin{array}{ll}
Z(\nu^k) &\mbox{\tiny for $k=0,\ldots,\frac{M}{4}$}\\
\quad & \quad\\
\overline{Z(\nu^{\frac{M}{2}-k})} &\mbox{\tiny for $k=\frac{M}{4}+1,\ldots,\frac{M}{2}-1$.}\\
\end{array} \right.
\end{eqnarray}
while if \dBP{\strA}{\strB} is odd, then
\begin{eqnarray}
\label{eq:trickyRootsOfUnity2}
y_k = Y(\omega^k) = \left\{
\begin{array}{ll}
\nu^{-k} \cdot Z(\nu^k) &\mbox{\tiny for $k=0,\ldots,\frac{M}{4}$}\\
\quad & \quad\\
-\nu^{-k} \cdot \overline{Z(\nu^{\frac{M}{2}-k})} &\mbox{\tiny for $k=\frac{M}{4}+1,\ldots,\frac{M}{2}-1$.}\\
\end{array} \right.
\end{eqnarray}
It follows that values $y_0,\ldots,y_{M/2-1}$ can be obtained by only
$\frac{M}{4}$ evaluations of $\Z(x)$.


\begin{figure}[!ht]
\begin{small}
{\sc Improved Algorithm} for \ffttwo\hfill\break
{\sc Input:} RNA sequence $\seq=s_1,\ldots,s_n$, and distinct secondary
structures $A,B$ of \seq, and integer $m$. \hfill\break
{\sc Output:} Probabilities $p(x,y)=\z{x,y}{1}{n}/{\bf Z}$
to $m$ significant digits for $x,y=0,\ldots,n-1$.
Let $N$ be the least even number greater than or equal
to $n$, $M$ be the least number evenly divisible by $4$, which is greater than
or equal to $(N+1)^2$, $M_0=M/2$, $\nu = \exp(\frac{2\pi i}{M})$,
$\omega=\nu^2= \exp(\frac{2\pi i}{M_0})$. For $0 \leq k < M^2$, let
$\pi_1(k) = \lfloor \frac{k}{M} \rfloor$,
$\pi_2(k) = k - M \cdot \pi_1(k) = k \bmod M$, and note that
$k=\pi_1(k)\cdot M + \pi_2(k)$.
\end{small}
\hfill\break
\smallskip
\begin{small}
\mverbatim
1.  for $k=0,\ldots,\frac{M}{2}$
2.    compute the $M$-roots of unity $\nu^k,\nu^{-k}$
3.  for $k=0,\ldots,\frac{M}{2}-1$
4.    if \dBP{\strA}{\strB} even
5.      if $k \leq \frac{M}{4}$
6.        $y_k = Y(\omega^k)=\Z(\nu^k)$ by (\ref{eq:trickyRootsOfUnity1})
7.      else// $\frac{M}{4}<k<\frac{M}{2}$
8.        $y_k = Y(\omega^k)=\overline{\Z(\nu^{M/2-k})}$ by (\ref{eq:trickyRootsOfUnity1})
9.    else // \dBP{\strA}{\strB} is odd
10.     if $k \leq \frac{M}{4}$
11.       $y_k = Y(\omega^k)=\nu^{-k} \cdot \Z(\nu^k)$ by (\ref{eq:trickyRootsOfUnity2})
12.     else// $\frac{M}{4}<k<\frac{M}{2}$
13.       $y_k = Y(\omega^k)= -1 \cdot \nu^{-k} \cdot \overline{\Z(\nu^{M/2-k})}$ by (\ref{eq:trickyRootsOfUnity2})
14. //note that ${\bf Z}=\sum_{r,s}\z{r,s}{1}{n}=y_0=\Z(\nu^0)$
15. for $k=0,\ldots,\frac{M}{2}-1$
16.   $y_k = 10^m \cdot \frac{y_k}{{\bf Z}}$ //normalize $y_k$


17. //compute coefficients of $\frac{Y(x)}{{\bf Z}}$ using (\ref{eq:ajdef})


18. if \dBP{\strA}{\strB} even then
19.   for $k=0$ to $M-1$
20.     $r=\pi_1(k)$, $s=\pi_2(k)$
21.     if $k$ even
22.       $\frac{\z{r,s}{1}{n}}{{\bf Z}} = a_{k/2}$ from (\ref{eq:ajdef})
23.     else// $k$ odd
24.       $\frac{\z{r,n}{1}{n}}{{\bf Z}} = 0$
25. else // \dBP{\strA}{\strB} odd
26.   for $k=0$ to $M-1$
27.     $r=\pi_1(k)$, $s=\pi_2(k)$
28.     if $k$ even
29.       $\frac{\z{r,n}{1}{n}}{{\bf Z}} = 0$
30.     else// $k$ odd
31.       $\frac{\z{r,n}{1}{n}}{{\bf Z}} = a_{(k-1)/2}$ from (\ref{eq:ajdef})
32.  for $k=0$ to $(N+1)^2$
33.    $z_k = \lfloor 10^m \cdot z_k \rfloor \cdot \frac{1}{10^m}$
34.    //truncate to $m$ significant digits
|mendverbatim
\end{small}
\caption{\small
Pseudocode to compute the $m$ most significant digits
for probabilities
$p_k = \frac{z_{k}}{{\bf Z}}=\frac{\z{\pi_1(k),\pi_2(k)}{1}{n}}{{\bf Z}}$.
Our program, \ffttwo, supports values of $m = 1,\ldots,8$ for the
precision parameter $m$.
(Note that the software actually usesbase $2$ precision parameter, with maximum of $27$, where $2^{27} \approx
10^8$.)
}
\label{pseudocode:interpolatingP}
\end{figure}


\subsection{Using the fast Fourier transform}
\label{section:FFT}

Now let $M_0=\frac{M}{2}$, let
$\nu=\exp(\frac{2\pi i}{M})$ be the principal $M$-root of unity, and
$\omega=\nu^2=\exp(\frac{2\pi i}{M/2})=\exp(\frac{2\pi \cdot 2i}{M})$ be
the principal $M_0$-root of unity. Recall that the
Vandermonde matrix $V_{M_0}$ is defined to be the
$M_0 \times M_0$ matrix, whose $i,j$ entry is
$\omega^{i \cdot j} = \nu^{2 i \cdot j}$;
i.e.
$$
V_{M_0} = \left(
\begin{array}{rrrrr}
1&1&1&\cdots&1\\
1&\omega&\omega^2&\cdots&\omega^{M_0-1}\\
1&\omega^2&\omega^4&\cdots&\omega^{2(M_0-1)}\\
1&\omega^3&\omega^6&\cdots&\omega^{3(M_0-1)}\\
\vdots& \vdots& \vdots& \vdots& \vdots\\
1&\omega^{M_0-1}&\omega^{2(M_0-1)}&\cdots&\omega^{(M_0-1)(M_0-1)}\\
\end{array}
\right)
$$
The Fast Fourier Transform (FFT) is the $O(n \log n)$
algorithm, which computes the Discrete Fourier Transform (DFT), defined
as the matrix product ${\bf Y} = V_{M_0} {\bf A}$:
$$
\left(
\begin{array}{l}
y_0\\
y_1\\
y_2\\
\vdots \\
y_{M_0-1}\\
\end{array}
\right)
= V_{M_0} \cdot
\left(
\begin{array}{l}
a_0\\
a_1\\
a_2\\
\vdots \\
a_{M_0-1}\\
\end{array}
\right)
$$
The $(i,j)$ entry of $V_{M_0}^{-1}$ is $\frac{\omega^{-j i}}{M_0}$
and that
\begin{eqnarray}
\label{eq:ajdef}
a_j &=&\frac{1}{M_0} \sum_{k=0}^{M_0-1} y_k \omega^{-kj}
=\frac{1}{M_0} \sum_{k=0}^{M_0-1} y_k \nu^{-2kj}
\end{eqnarray}
for $j=0,\ldots,M_0-1$ (for more on FFT, see \citep{cormen}).

Since we defined $\bf Y$ in (\ref{eq:defY}) by ${\bf Y} =
(y_0,\ldots,y_{M_0-1})^T$, where
$y_0=\Z(\alpha_0),\ldots, y_{M_0-1}=\Z(\alpha_{M_0-1})$ and $\alpha_k = \omega^k
\exp(\frac{k \cdot 2\pi i}{M_0})$,
it follows that the coefficients
$z_k=\z{\pi_1(k),\pi_2(k)}{1}{n}$ in the polynomial
$\Z(x) = z_0 + z_1 x + \cdots + z_{M} x^{M}$ defined in
(\ref{eq:polynomialYann}) can be computed, at least in principle,
by using the FFT. However, since the values of
$z_{k}$ are astronomically large, numerical
instability makes even this approach infeasible for moderate values of $n$.
Nevertheless, we apply this approach to compute the $m$ most significant
digits of $\frac{\z{\pi_1(k),\pi_2(k)}{1}{n}}{{\bf Z}}$,
where the partition function ${\bf Z} = \sum_{S} \exp(-E(S)/RT)$ satisfies
${{\bf Z}} = \sum_{x,y} \z{x,y}{1}{n}$. This leads to numerical stability,
allowing \ffttwo to compute the
$m$ most significant digits of $p(x,y) = \frac{\z{x,y}{1}{n}}{{\bf Z}}$.
Pseudocode for the complete algorithm, \fftbor, is given in
%Figure \ref{pseudocode:interpolatingP}.
Figure 2.


\section{Benchmarking}
\label{section:benchmarking}

To perform comparative benchmarking between \rnatwofold and \ffttwo,
we took precision parameter $m=8$, and proceeded as follows.
For each sequence length $n = 20,25,30,\ldots,300$, we generated
$100$ random sequences using probability $0.25$ for each nucleotide A,C,G,U.
For a given RNA sequence \seq, the metastable structure \strA was
taken to be the MFE structure of \seq.
Using {\tt RNAbor}, we determined that value $k_0\geq 10$, for which
partition function $Z_{k_0}$ constitutes a visible peak in the graphical
output -- see Figure 2 and 3 of  \citep{Freyhult.b07} for an example.
Subsequently, metastable structure \strB was taken to be that structure
having minimum free energy over all structures, whose base pair distance from
\strA was $k_0$.

For all $0 \leq x,y \leq n$, \rnatwofold and \ffttwo were
benchmarked in the computation of all Boltzmann probabilities
$p(x,y)= \frac{Z(x,y)}{Z}$, where $x$ [resp. $y$] represents base pair
distance to metastable structure \strA [resp. \strB]. Care was taken for
both software to employ the same energy model (Turner99 energy model,
no dangles, suppression of minimum free energy structure computations
for \rnatwofold) and the same number of parallel threads (8 threads
using OpenMP). Nonetheless, there are slight differences in the energy
models -- namely, \rnatwofold includes
mismatch penalties for multiloop stems and for exterior loops, while
\ffttwo does not. Even in the computation of the partition function
$Z$, for spliced leader RNA from {\em L. collosoma} of length 56 nt,
\rnatwofold {\tt -d0} obtains a value of
$-9.660419$ kcal/mol, while \ffttwo obtains
$-9.660543$ kcal/mol; similarly, for attenuator RNA of length 73 nt,
\rnatwofold {\tt -d0} obtains a value of
$-22.171785$ kcal/mol, while \ffttwo obtains
$-22.173213$ kcal/mol.
Note that the straightforward calculation of the partition function,following McCaskill's algorithm \citep{mcCaskill}makes no use of the FFT engine, and thus the
differences cannot be due to floating point or precision issues.

For benchmarking purposes,
to allow for a fair comparison of \ffttwo
with \rnatwofold, we restricted therange of $x,y$ in the same manner as done in the source code of \rnatwofold.
In that code, parameters $K$ [resp. $L$] are defined respectively to be the
sum of the number of base pairs in reference structure \strA
[resp. reference structure \strB] plus the number of base pairs in
the maximum matching (Nussinov) structure which contains
no base pair of \strA [resp. \strB].
For $x \geq K, y \geq L$, both \rnatwofold and \ffttwo
set $p(x,y)=0$.  For the benchmarking results displayed in
%Figures~\ref{fig:benchmarking1}, \ref{fig:benchmarking2},
%\ref{fig:benchmarking3}, the values $x,y$ are restricted in
Figures~3,4,5, the values $x,y$ are restricted in
\ffttwo to $0 \leq x,y \leq \max(K,L)$, while $0 \leq x \leq K$
and $0 \leq y \leq L$ in \rnatwofold.

%Figure \ref{fig:benchmarking1} depicts average run time of \rnatwofold
Figure 3 depicts average run time of \rnatwofold
and \ffttwo as a function of RNA sequence length, for random RNA
sequences of lengths $20-200$ and their metastable structures $A,B$, as
previously explained.  We see that both programs have
roughly comparable run times for sequences of length up to approximately
80 nt, while \ffttwo is demonstrably faster for longer sequences.
%Figure \ref{fig:benchmarking2} is an enlargement of
Figure 4 presents log run time as a function of sequence length, in order
to more clearly determine the crossover point in performance.
\rnatwofold is marginally faster for sequences of
length up to roughly 80 nt, though the difference is in the millisecond
range.
%Figure \ref{fig:benchmarking3}
Figure 5
shows that the standard deviation of run times on random sequences is
tiny for \ffttwo compared with \rnatwofold, where standard
deviation increases rapidly as a function of sequence length. This figure
shows that run time of \rnatwofold depends on sequence details, as
well as sequence length, while the run time of \ffttwo depends only
on sequence length.

%\begin{figure}[!bpth]
%\begin{figure}[!thbp]
\begin{figure}[!t]
\begin{center}
\includegraphics[width=.8\textwidth]{FIGURES/time_benchmarking_fftbor2d_vs_rna2dfold}
\caption{\small
Run time in seconds for \rnatwofold and \ffttwo on random
RNA sequences of length $20-200$ nt, where sequence generation and
choice of metastable structures $A,B$ is described in the text.
Beyond a length of approximately $80$ nt, \ffttwo is demonstrably
faster.
}
\label{fig:benchmarking1}
\end{center}
\end{figure}


%\begin{figure}[!bpth]
\begin{figure}[!b]
\begin{center}
\includegraphics[width=.8\textwidth]{FIGURES/time_benchmarking_fftbor2d_vs_rna2dfold_log_scale}
\caption{\small
Logarithm of run time in seconds for \rnatwofold and \ffttwo
on random RNA sequences of length less than $200$ nt, for same data as that
%in Figure \ref{fig:benchmarking1}. By taking logarithm of run times, the
in Figure 3. By taking logarithm of run times, the
crossover points are apparent,
where \ffttwo is faster than \rnatwofold. For very small
sequences, \rnatwofold is faster, though since both programs converge
in a fraction of a second, this difference is of no practical consequence.
}
\label{fig:benchmarking2}
\end{center}
\end{figure}

\begin{figure}[!bpth]
\begin{center}
\includegraphics[width=.45\textwidth]{FIGURES/time_benchmarking_fftbor2d_vs_rna2dfold_stdev_runtimes}
\includegraphics[width=.45\textwidth]{FIGURES/time_min_max_benchmarking_fftbor2d_vs_rna2dfold}
\caption{\small
{\em (Left)}
Standard deviation of run times of \rnatwofold and \ffttwo
as a function of sequence length $n$.
{\em (Right)}
Minimum and maximum run times for \rnatwofold and \ffttwo.
For each collection of 100 random sequences of length $n$, the minimum
and maximum run time for a sequence of that length was computed.
Taken together, these figures clearly show the
run time dependence of \rnatwofold on particular sequences, while
the run time of \ffttwo depends only on sequence length, rather than
sequence details.
}
\end{center}
\label{fig:benchmarking3}
\end{figure}


%\begin{figure}[!tbph]
\begin{figure}[!t]
\begin{center}
\includegraphics[width=.50\textwidth]{FIGURES/PeterFFTbor2DLcollosoma}
\hskip 1cm
\includegraphics[width=.35\textwidth]{FIGURES/FFTbor2D_prob_contour}
\caption{\small
2D projection of energy landscape for Spliced Leader (SL) RNA
from {\em Leptomonas collosoma}, having
sequence
{\tt AACUAAAACA AUUUUUGAAG AACAGUUUCU GUACUUCAUU GGUAUGUAGA GACUUC} and
metastable secondary structures  $A=$
{\tiny \tt ..((...((((((..(((((.((((...)))).)))))..))).)))..)).....}, and
$B=$ {\tiny \tt .......................((((((((((((.....)))))..)))))))..}.
The $x$-axis (resp. $y$-axis)
represents base pair distance between metastable structure \strA (resp. \strB),
while the $z$-axis represents the ensemble free energy $-RT \log Z_{x,y}$,
where $Z_{x,y}$ is computed in \ffttwo by $Z_{x,y}= p(x,y) \cdot Z$.
Low energy positions $(x,y)$ correspond to high Boltzmann probability positions.
The left panel depicts a heat map of the ensemble free energy,
while the right panel depicts a contour map with level curves. In analogy
with mountain climbing, one expects an optimal path to follow along the
valley regions in traversing the landscape from \strA to \strB. Data produced
with \ffttwo; graphics produced using Mathematica.
}
\label{fig:heatmapFFTbor2D}
\end{center}
\end{figure}





\section{Kinetics}
\label{section:kinetics}

In this section, we describe folding kinetics along the 2D energy
grid, as depicted in
%Figure \ref{fig:heatmapFFTbor2D}.
Figure 6.
Consider the
56 nt {\em L. collosoma} spliced leader RNA \citep{lecuyerCrothers},
described in the Introduction, having
sequence AACUAAAACA AUUUUUGAAG AACAGUUUCU GUACUUCAUU GGUAUGUAGA
GACUUC.
Let \strA denote the minimum free energy structure of spliced leader,
using Turner 1999 energies as implemented in Vienna RNA Package 1.8.5:
\begin{quote}
{\tt ..((...((((((..(((((.((((...)))).)))))..))).)))..)).....},
\end{quote}
and let \strB denote the low energy alternate structure for spliced leader:
\begin{quote}
{\tt .......................((((((((((((.....)))))..)))))))..}.
\end{quote}
Using the program {\em Switch Design} ({\tt switch.pl}) described in
\citep{Flamm.r01}, we generated 20,000 sequences, for which structures
$A,B$ are metastable. For spliced leader RNA as well as for each of
these 20,000 sequences, we computed the corresponding
probability profile $p(x,y)$ using \ffttwo,
and subsequently defined the Markov chain ${\mathbb{M}}(\seq) = (Q,M)$, where
$Q = \{ (x,y) : 0 \leq x,y \leq n, \mbox{ and } p(x,y)>0 \}$
is the set of states, and the
transition probability matrix $M = (M_{i,j})$ is defined by
\begin{eqnarray*}
M_{(x,y), (u,v)} = \left\{ \begin{array}{ll}
\frac{1}{|Q|-1} \cdot \min(1, \frac{p(u,v)}{p(x,y)})
&\mbox{if $(u,v) \ne (x,y)$}\\
%1 - \sum_{\substack{(u,v) \in Nbor(x,y)}} M_{(x,y),(u,v)}
1 - \displaystyle\sum_{\substack{(u,v) \ne (x,y)}} M_{(x,y),(u,v)}
&\mbox{if $(u,v) = (x,y)$. }\\
\end{array} \right.
\end{eqnarray*}
Let $d_0=\dBP{\strA}{\strB}$ denote the base pair distance between the metastable
structures $A,B$, and let $M^{-}_{(d_0,0)}$ denote the matrix obtained from
$M$ by removing both the and column corresponding to $(0,d_0)$.
For spliced leader and each of the 20,000 sequences obtained from
{\em Switch Design}, we determined the {\em mean first passage time} (MFPT)
from state $(0,d_0)$, corresponding to metastable structure \strA,
 to state $(d_0,0)$, corresponding to metastable structure \strB,  by
computing $(I - M^{-}_{(d_0,0)})^{-1} \cdot {\bf e}$, where
$I$ denotes the identity matrix, and ${\bf e}$ denotes the column vector
composed entirely of ones \citep{meyerMFPT}. Using LAPACK \citep{LAPACK}
for matrix inversion, we found that
{\em L. collosoma} spliced leader RNA has a
MFPT on the 2D energy grid, which is smaller than only $2.855\%$ of the
20,000 sequnces generated by {\em Switch Design}, thus constituting a
$Z$-score of $1.989$ for the kinetics of folding from \strA to \strB
%(see left panel of Figure \ref{fig:histogramMFPT}).
(see left panel of Figure 7).
This result seems to suggest that spliced leader could be
under evolutionary pressure for {\em slow} folding between these
metastable structures, if we take
MFPT from $(0,d_0)$ to $(d_0,0)$ as a surrogate for \kinfold \citep{flamm}
folding time from \strA to \strB -- an interpretation which seems to be
consistent with the functional role of spliced leader as described
in the Introduction.
Since accurate \kinfold kinetics requires many simulations,
each requiring enormous time \citep{wolfingerStadler:kinetics},
our method may prove useful in synthetic biology, in prioritizing
computationally designed RNA sequences for subsequent
experimental validation.

Finally, it should be mentioned that the GC-content of spliced leader
RNA is
%$0.30357142857142855$
$30.357\%$, which constitutes a $Z$-score of
%2.500575245690564
$+2.50$; i.e. the overwhelming majority of the 20,000 sequences generated
by {\em Switch Design} have higher GC-content than that of spliced leader
RNA from {\em L. collosoma}, as shown in the right panel of
%Figure \ref{fig:histogramMFPT}.
Figure 7.

\begin{figure}[!tbph]
\begin{center}
%\includegraphics[width=.8\textwidth]{FIGURES/mfptsRelativeFreqsArrow}
%\includegraphics[width=.8\textwidth]{FIGURES/log_histogram_for_switch_design_using_fftbor}
\includegraphics[width=.45\textwidth]{FIGURES/log_histogram_for_switch_design_using_fftbor}
\includegraphics[width=.45\textwidth]{FIGURES/gc_content_histogram_for_real_spliced_leader_vs_switch_design}
\caption{\small
{\em (Left)}
Histogram of log base 10 mean first passage times,
computed by $(I - M^{-}_{(d_0,0)})^{-1} \cdot {\bf e}$ (see text),
for a collection of 20,000 RNA sequences \seq, each of has a metastable
structure at the minimum free energy structure \strA of
{\em L. collosoma} spliced leader RNA, given by
{\tiny \tt ..((...((((((..(((((.((((...)))).)))))..))).)))..)).....},
as well as a metastable structure at the alternate structure \strB,
given by
{\tiny \tt .......................((((((((((((.....)))))..)))))))..}.
These sequences were generated by the program {\em Switch Design}
({\tt switch.pl}) described in \citep{Flamm.r01}, using the Turner 1999
energy model without dangles.
The log base 10 value of MFPT of {\em L. collosoma} spliced leader
is indicated by the arrow in figure, corresponding to a $Z$-score of
%$1.98907157392517831$, which suggests
$1.989$, which suggests
that spliced leader RNA may be under evolutionary pressure
for slow folding kinetics from \strA to \strB.
{\em (Right)} Histogram of GC-content of the 20,000 sequences generated
by {\em Switch Design}. Note that GC-content of spliced leader RNA is
%$0.30357142857142855$ (exact value)
$30.357\%$, which constitutes a $Z$-score of
%2.500575245690564 (exact value)
$+2.50$.
}
\end{center}
\label{fig:histogramMFPT}
\end{figure}

%for reasons of time, we have not yet looked into this possibility.
%It may also be possible to replace our current use of matrix inversion in
%computing MFPT by the recent celebrated method
%of Condmain et al. \citep{Condamin.n07}.


\section{Discussion}
\label{section:discussion}

Given an RNA sequence \seq and two reference secondary structures $A,B$,
the algorithm, \ffttwo, computes the partition
function $Z(x,y)$, defined to be the sum of Boltzmann factors
$\exp(-E(S)/RT)$ over all secondary structures $S$, having base pair
distance $x$ to  \strA and distance $y$ to \strB, where
$0 \leq x,y \leq n$ and $n$ denotes the length \seq.
Using polynomial interpolation with the FFT and exploiting the observations
of Lemmas 1,2, \ffttwo has worst case complexity
$O(n^5)$ time and $O(n^2)$ space. This worst case
algorithmic complexity is two orders of magnitude faster and requires two
orders of magnitude less space than the worst case complexity of the
algorithm \rnatwofold of Lorenz et al. \citep{hofacker:RNAbor2D}.
This run time complexity bound is not only theoretical, but entails
a significant practical speedup, as depicted in
Figures 3,4 and 5.
%Figures~\ref{fig:benchmarking3}, \ref{fig:benchmarking4}, and
%\ref{fig:benchmarking5}.

An important advantage of
\rnatwofold over \ffttwo is that the former can additionally
compute the structures $M_{x,y}$ having minimum free energy over all
structures that are $x$-neighbors of metastable \strA and simultaneously
$y$-neighbors of metastable \strB. (There is a similar advantage of {\tt RNAbor}
\citep{Freyhult.b07} over the faster \fftbor \citep{fftbor}.)
As well, \rnatwofold directly computes the partition function values
$Z_{x,y}$, while \ffttwo estimates $Z_{x,y}$ by computing
$p(x,y) \cdot Z$. This difference entails a significant loss of precision,
when depicting the energy landscape.


%\begin{figure}[!tbph]
\begin{figure}[!t]
\begin{center}
\includegraphics[width=.35\textwidth]{FIGURES/RNA2Dfold_ensFreeEnergy_Lcollosoma}
\hskip 1cm
\includegraphics[width=.35\textwidth]{FIGURES/RNA2Dfold_prob_Lcollosoma}
\caption{\small
2D projection of energy landscape for Spliced Leader (SL) RNA
from {\em Leptomonas collosoma}, as in
%Figure \ref{fig:heatmapFFTbor2D}
Figure 6,
except that in the left panel, ensemble free energy $-RT \log Z_{x,y}$
is computed from the values of $Z_{x,y}$ output by \rnatwofold,
while in the right panel, ensemble free energy is computed from
the values $Z_{x,y} = p(x,y) \cdot Z$, where values $p(x,y)$ are output
by \rnatwofold.
The loss of detail in the 2D energy landscape is caused uniquely by
working with probabilities $p(x,y)$, rather than partition function
values $Z_{x,y}$.
Data produced with \rnatwofold; graphics produced using Mathematica.
}
\label{fig:heatmapRNA2Dfold}
\end{center}
\end{figure}

%The right panel of Figure \ref{fig:heatmapFFTbor2D} depicts a contour
The right panel of Figure 6 depicts a contour
heat map of the 2D energy landscape for spliced leader RNA from
{\em L. collosoma}, as computed by \ffttwo.
This figure should be compared with the left panel of
%Figure \ref{fig:heatmapRNA2Dfold}
Figure 8, which depicts a contour heat map of the 2D energy landscape for
the same RNA, as computed by \rnatwofold.
Notice the additional detail in this figure, due
to the fact that \rnatwofold directly computes $Z_{x,y}$, while
\ffttwo computes Boltzmann probabilities $p(x,y)$ by interpolation,
allowing one to subsequently compute $Z_{x,y} = p(x,y) \cdot Z$.
The additional detail of the energy landscape is lost in the right panel of
%Figure \ref{fig:heatmapRNA2Dfold}
Figure 8, obtained by computing ensemble free energy by
$-RT \log p(x,y) + RT \log cdot Z$, where $p(x,y)$ is parsed from
\rnatwofold output.  It follows that the loss of detail in
2D energy landscape is due solely to the fact that probabilities $p(x,y)$ are
computed by \ffttwo, rather than partition function values
$Z_{x,y}$. Given numerical stability issues involving the FFT engine,
\ffttwo can only estimate the probabilities $p(x,y)$ to within
$m=8$ significant places. Nevertheless, our algorithm \ffttwo
was developed with the intended application in synthetic biology, where
one wishes to prioritize RNA candidate sequences with respect to kinetics.
For such applications, the speedup of \ffttwo is an important asset.
